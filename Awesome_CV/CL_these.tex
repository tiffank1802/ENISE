% !TEX TS-program = xelatex
\documentclass[11pt, a4paper]{awesome-cv}

\geometry{left=2cm, top=2cm, right=2cm, bottom=2cm, footskip=.5cm}
\fontdir[fonts/]

\colorlet{awesome}{awesome-red}
\setbool{acvSectionColorHighlight}{true}
\renewcommand{\acvHeaderSocialSep}{\quad\textbar\quad}

%-------------------------------------------------------------------------------
%   PERSONAL INFORMATION
%-------------------------------------------------------------------------------
\name{Kevin}{Tongue}
\position{5th Year Student -- Mechanical Engineering}
\address{65, Rue Jean Parot, 42000 Saint-Étienne, France}

\mobile{(+33) 6 98 06 37 69}
\email{tonguekevin00@gmail.com}
\github{tittank1802}
\linkedin{tongue-kevin-52b100330}

%-------------------------------------------------------------------------------
%   LETTER INFORMATION
%-------------------------------------------------------------------------------
\recipient
  {Selection Committee\\MAD-SIM PhD Position}
  {IMT Mines Albi\\Campus Jarlard\\81013 Albi, France}
\date{\today}
\opening{Dear Dr. Gatumel, Dr. Hämäläinen, and Members of the Selection Committee,}
\closing{Yours sincerely,}
\enclosure{Curriculum Vitae, Academic Transcripts, Recommendation Letters}

\begin{document}
\makecvheader
\makelettertitle

\begin{cvletter}

\lettersection{Subject: Application for the MAD-SIM PhD Position -- Modelling Powder Flows in Agitated Devices}

As a \textbf{5th-year Mechanical Engineering student} at Centrale Lyon -- ENISE, specializing in \textbf{Numerical Solid Mechanics}, I am writing to express my strong interest in the PhD position titled \emph{``Modelling powder flows in agitated devices: from rheological studies to predictive SIMulation at process scale''}, offered by IMT Mines Albi in collaboration with LUT and École des Mines de Saint-Étienne. I am particularly motivated to apply as I have been selected for a preparatory internship on granular segregation modeling starting in March 2026, which will directly introduce me to the research themes of this PhD project.

\lettersection{Direct Pathway: From Internship to PhD}

I am pleased to inform you that I have been accepted for a 6-month internship at IMT Mines Albi and IMT Mines Saint-Étienne, titled \emph{``Modeling of Granular Segregation Phenomena using Inhomogeneous Markov Chains''}, which will begin in March 2026. This internship will focus on using \textbf{Discrete Element Method (DEM) simulations} to analyze granular segregation in mixing processes and developing \textbf{stochastic models} using inhomogeneous Markov chains.

This internship represents an ideal transition to the MAD-SIM PhD project, as it will allow me to:
\begin{itemize}
  \item Gain hands-on experience with \textbf{granular flow simulation techniques} (DEM)
  \item Develop expertise in \textbf{stochastic modeling} applied to granular materials
  \item Familiarize myself with the research environment and teams at IMT Mines Albi and École des Mines de Saint-Étienne
  \item Build a solid foundation in granular physics that will directly support the PhD research objectives
\end{itemize}

By the time the PhD begins in October 2026, I will already have substantial experience in granular flow modeling, making me particularly well-prepared to contribute immediately to the MAD-SIM project.

\lettersection{Interest in Powder Flow Modeling}

The challenge of \textbf{modeling powder flows in agitated devices} lies at the intersection of several fields that deeply motivate me: granular mechanics, rheology, numerical simulation, and industrial applications. The opportunity to bridge \textbf{local rheological studies} with \textbf{predictive continuous models at process scale}, using both experimental and simulation approaches (DEM, CFD), aligns perfectly with the type of complex problems I aspire to tackle during my PhD.

This project’s goal of \textbf{linking experimental observations on an instrumented pilot} to multi-scale numerical models, in order to propose rheological laws adapted to real powders and implement them in industrial-scale simulation tools, resonates strongly with my professional ambition: contributing to the development of \textbf{digital twins} for processes and optimizing equipment in industries such as pharmaceuticals, food processing, and energy.

\lettersection{Alignment of My Profile with the PhD Objectives}

Throughout my academic journey, I have developed a strong foundation in \textbf{solid mechanics} and \textbf{numerical methods}. My current project on \textbf{fatigue analysis of industrial structures under complex loading conditions} involves validating advanced numerical methods (Agard et al.) for optimizing the Dang Van method, which has strengthened my ability to \textbf{analyze, implement, and validate mechanical models} in an industrial context.

Additionally, my Python-based project on \textbf{microstructure partitioning}, conducted at École des Mines de Saint-Étienne, allowed me to develop skills in \textbf{scientific programming}, data processing, and material modeling. These skills are directly transferable to analyzing velocity or stress fields from DEM or CFD simulations and identifying effective laws (e.g., effective viscosity or $\mu(I)$ laws) from local data.

My expertise in \textbf{programming} (Python, Matlab, C/C++, Fortran) and my \textbf{Matlab certifications} (Simulink, Computer Vision, Power System) demonstrate my proficiency with scientific computing and simulation tools. These skills will enable me to fully engage in the development and exploitation of numerical models, whether for \textbf{local DEM simulations} or \textbf{continuous models implemented in CFD codes}.

\lettersection{Motivation for an International and Interdisciplinary Environment}

The \textbf{international} and \textbf{interdisciplinary} nature of this PhD is particularly appealing to me. The co-supervision between IMT Mines Albi, LUT, and École des Mines de Saint-Étienne, along with the possibility of obtaining a dual French-Finnish doctorate, represents a unique opportunity to immerse myself in complementary scientific environments and develop a \textbf{broad perspective on granular process challenges}.

I am especially motivated by the prospect of working within the \textbf{RAPSODEE research center} in Albi, renowned for its expertise in particulate solids process engineering, while benefiting from the advanced numerical methods expertise of the \textbf{LUT School of Engineering Sciences} and the academic environment of \textbf{École des Mines de Saint-Étienne}. This setup aligns perfectly with my aspiration to build a career at the interface of \textbf{academic research} and \textbf{industrial applications}.

\lettersection{Conclusion}

In summary, my upcoming internship on granular segregation modeling, combined with my background in \textbf{numerical mechanics}, experience in \textbf{modeling and scientific programming}, and motivation for \textbf{granular flows} and \textbf{industrial processes}, makes me a particularly strong candidate for the MAD-SIM PhD position. I am eager to begin my preparatory internship in March 2026 and then fully commit to this ambitious PhD project, both experimentally and numerically, contributing to the development of innovative simulation and characterization tools for powder flows in agitated devices.

Thank you for considering my application. I am available for an interview or to provide any additional information you may require.

\end{cvletter}

\makeletterclosing

\end{document}
