%!TEX TS-program = xelatex
%!TEX encoding = UTF-8 Unicode
% Awesome CV LaTeX Template for Cover Letter
%
% This template has been downloaded from:
% https://github.com/posquit0/Awesome-CV
%
% Authors:
% Claud D. Park <posquit0.bj@gmail.com>
% Lars Richter <mail@ayeks.de>
%
% Template license:
% CC BY-SA 4.0<a href="https://creativecommons.org/licenses/by-sa/4.0/" target="_blank" rel="noopener noreferrer nofollow"></a>
%
%-------------------------------------------------------------------------------
% CONFIGURATIONS
%-------------------------------------------------------------------------------
% A4 paper size by default, use 'letterpaper' for US letter
\documentclass[11pt, a4paper]{awesome-cv}

% Configure page margins with geometry
\geometry{left=1.4cm, top=.8cm, right=1.4cm, bottom=1.8cm, footskip=.5cm}

% Specify the location of the included fonts
\fontdir[fonts/]

% Color for highlights
% Awesome Colors: awesome-emerald, awesome-skyblue, awesome-red, awesome-pink, awesome-orange
% awesome-nephritis, awesome-concrete, awesome-darknight
\colorlet{awesome}{awesome-red}
% Uncomment if you would like to specify your own color
% \definecolor{awesome}{HTML}{CA63A8}

% Colors for text
% Uncomment if you would like to specify your own color
% \definecolor{darktext}{HTML}{414141}
% \definecolor{text}{HTML}{333333}
% \definecolor{graytext}{HTML}{5D5D5D}
% \definecolor{lighttext}{HTML}{999999}

% Set false if you don't want to highlight section with awesome color
\setbool{acvSectionColorHighlight}{true}

% If you would like to change the social information separator from a pipe (|) to something else
\renewcommand{\acvHeaderSocialSep}{\quad\textbar\quad}

%-------------------------------------------------------------------------------
% PERSONAL INFORMATION
% Comment any of the lines below if they are not required
%-------------------------------------------------------------------------------
% Available options: circle|rectangle,edge/noedge,left/right
\photo[circle,noedge,left]{./examples/profile}
\name{Kevin}{TONGUE}
\position{5th year student{\enskip\cdotp\enskip}Mechanical engineer}
\address{65, Rue Jean Parot , Saint-Étienne,France}

\mobile{(+33) 06 98 06 37 69}
\email{tonguekevin00@gmail.com}
% \homepage{www.posquit0.com}
\github{tittank1802}
\linkedin{tongue-kevin-52b100330}
% \gitlab{gitlab-id}
% \stackoverflow{SO-id}{SO-name}
% \twitter{@twit}
% \skype{skype-id}
% \reddit{reddit-id}
% \medium{madium-id}
% \googlescholar{googlescholar-id}{name-to-display}
%% \firstname and \lastname will be used
% \googlescholar{googlescholar-id}{}
% \extrainfo{extra informations}

% \quote{``Be the change that you want to see in the world."}

%-------------------------------------------------------------------------------
% LETTER INFORMATION
% All of the below lines must be filled out
%-------------------------------------------------------------------------------
% The company being applied to
\recipient{Recruitment Team}{LAMIH UMR CNRS 8201 (in collaboration with Framatome)}

% The date on the letter, default is the date of compilation
\letterdate{\today}

% The title of the letter
\lettertitle{Application for Internship: Ingénieur Modélisation du Soudage}

% How the letter is opened
\letteropening{Dear Yabo Jia and Recruitment Team,}

% How the letter is closed
\letterclosing{Sincerely,}

% Any enclosures with the letter
% \letterenclosure[Attached]{Curriculum Vitae}

%-------------------------------------------------------------------------------
\begin{document}

% Print the header with above personal informations
% Give optional argument to change alignment(C: center, L: left, R: right)
\makecvheader[R]

% Print the footer with 3 arguments(<left>, <center>, <right>)
% Leave any of these blank if they are not needed
\makecvfooter
  {\today}
  {Kevin TONGUE~~~·~~~Cover Letter}
  {}

% Print the title with above letter informations
\makelettertitle

%-------------------------------------------------------------------------------
% LETTER CONTENT
%-------------------------------------------------------------------------------
\begin{cvletter}

\lettersection{About Me}
As a fifth-year Mechanical Engineering student specializing in numerical simulation at Centrale Lyon-ENISE, I am writing to express my strong interest in the internship position as "Ingénieur Modélisation du Soudage" at LAMIH UMR CNRS 8201 in collaboration with Framatome. My academic background in numerical solid mechanics, combined with my experience in simulation tools and finite element methods, aligns perfectly with the requirements of this project focused on welding process modeling.

\lettersection{Why This Internship?}
The challenge of modeling welding processes, including heat transfer and residual stress analysis, represents exactly the type of complex mechanical simulation problem that drives my passion for engineering. This opportunity particularly appeals to me for several reasons:
• The project combines advanced numerical simulation with practical industrial applications in nuclear and energy sectors through collaboration with Framatome
• It involves finite element methods for multi-physics phenomena like thermal-mechanical coupling in welding
• The focus on residual stress and heat transfer simulations represents a cutting-edge approach to improving manufacturing reliability
• The possibility to contribute to innovative research in materials mechanics aligns with my career aspirations in advanced simulation

The research environment at LAMIH UMR CNRS 8201 in Valenciennes provides an ideal setting for professional growth in numerical mechanics.

\lettersection{Why My Profile Fits?}
Throughout my academic journey, I have developed a strong foundation in numerical methods and simulation that directly applies to this internship:
• \textbf{Numerical Simulation and Finite Elements:} My Master's specialization in Numerical Solid Mechanics has equipped me with advanced knowledge of finite element methods (FEM) for solving thermal and mechanical problems, including stress analysis in materials
• \textbf{Heat Transfer and Stress Modeling:} In my fatigue analysis project at Centrale Lyon ENISE, I validated numerical methods for complex loads, involving thermal-mechanical simulations and critical plane determination, directly relevant to welding residual stresses
• \textbf{Programming Skills:} I have substantial experience with Python for developing models (e.g., microstructure partitioning) and MATLAB for scientific computing, including certifications in Simulink and Power System OnRamp for multi-physics simulations
• \textbf{Materials Mechanics Understanding:} My background in mechanical engineering and projects like microstructure analysis provide the foundation to understand welding processes and material behavior under thermal loads
• \textbf{Research Experience:} My work on numerical validation and optimization has prepared me for autonomous research, with strong analytical and synthesis skills

My hands-on experience in FEM-based simulations, combined with my programming expertise and mechanical background, positions me well to contribute to the modeling of welding processes, heat transfers, and residual stresses in this internship.

I am particularly excited about the prospect of working on a project with direct industrial impact through Framatome collaboration. The opportunity to apply my skills in a real-world nuclear engineering context is highly appealing.

Thank you for considering my application. I am confident that my technical skills and research capabilities align well with your requirements, and I would welcome the opportunity to contribute to your team's work on welding simulation. I have attached my CV, cover letter, and transcripts from the last two years, and can provide a recommendation letter upon request.

\end{cvletter}

%-------------------------------------------------------------------------------
% Print the signature and enclosures with above letter informations
\makeletterclosing

\end{document}