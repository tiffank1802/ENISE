%!TEX TS-program = xelatex
%!TEX encoding = UTF-8 Unicode
% Awesome CV LaTeX Template for Cover Letter
%
% This template has been downloaded from:
% https://github.com/posquit0/Awesome-CV
%
% Authors:
% Claud D. Park <posquit0.bj@gmail.com>
% Lars Richter <mail@ayeks.de>
%
% Template license:
% CC BY-SA 4.0<a href="https://creativecommons.org/licenses/by-sa/4.0/" target="_blank" rel="noopener noreferrer nofollow"></a>
%
%-------------------------------------------------------------------------------
% CONFIGURATIONS
%-------------------------------------------------------------------------------
% A4 paper size by default, use 'letterpaper' for US letter
\documentclass[11pt, a4paper]{awesome-cv}

% Configure page margins with geometry
\geometry{left=1.4cm, top=.8cm, right=1.4cm, bottom=1.8cm, footskip=.5cm}

% Specify the location of the included fonts
\fontdir[fonts/]

% Color for highlights
% Awesome Colors: awesome-emerald, awesome-skyblue, awesome-red, awesome-pink, awesome-orange
% awesome-nephritis, awesome-concrete, awesome-darknight
\colorlet{awesome}{awesome-red}
% Uncomment if you would like to specify your own color
% \definecolor{awesome}{HTML}{CA63A8}

% Colors for text
% Uncomment if you would like to specify your own color
% \definecolor{darktext}{HTML}{414141}
% \definecolor{text}{HTML}{333333}
% \definecolor{graytext}{HTML}{5D5D5D}
% \definecolor{lighttext}{HTML}{999999}

% Set false if you don't want to highlight section with awesome color
\setbool{acvSectionColorHighlight}{true}

% If you would like to change the social information separator from a pipe (|) to something else
\renewcommand{\acvHeaderSocialSep}{\quad\textbar\quad}

%-------------------------------------------------------------------------------
% PERSONAL INFORMATION
% Comment any of the lines below if they are not required
%-------------------------------------------------------------------------------
% Available options: circle|rectangle,edge/noedge,left/right
\photo[circle,noedge,left]{./examples/profile}
\name{Kevin}{TONGUE}
\position{5th year student{\enskip\cdotp\enskip}Mechanical engineer}
\address{65, Rue Jean Parot , Saint-Étienne,France}

\mobile{(+33) 06 98 06 37 69}
\email{tonguekevin00@gmail.com}
% \homepage{www.posquit0.com}
\github{tittank1802}
\linkedin{tongue-kevin-52b100330}
% \gitlab{gitlab-id}
% \stackoverflow{SO-id}{SO-name}
% \twitter{@twit}
% \skype{skype-id}
% \reddit{reddit-id}
% \medium{madium-id}
% \googlescholar{googlescholar-id}{name-to-display}
%% \firstname and \lastname will be used
% \googlescholar{googlescholar-id}{}
% \extrainfo{extra informations}

% \quote{``Be the change that you want to see in the world."}

%-------------------------------------------------------------------------------
% LETTER INFORMATION
% All of the below lines must be filled out
%-------------------------------------------------------------------------------
% The company being applied to
\recipient{Recruitment Team}{CESI LINEACT}

% The date on the letter, default is the date of compilation
\letterdate{\today}

% The title of the letter
\lettertitle{Application for M2 Internship: Multimodal Approaches for Handling Missing Information in Affective States Analysis}

% How the letter is opened
\letteropening{Dear Recruitment Team,}

% How the letter is closed
\letterclosing{Sincerely,}

% Any enclosures with the letter
% \letterenclosure[Attached]{Curriculum Vitae}

%-------------------------------------------------------------------------------
\begin{document}

% Print the header with above personal informations
% Give optional argument to change alignment(C: center, L: left, R: right)
\makecvheader[R]

% Print the footer with 3 arguments(<left>, <center>, <right>)
% Leave any of these blank if they are not needed
\makecvfooter
  {\today}
  {Kevin TONGUE~~~·~~~Cover Letter}
  {}

% Print the title with above letter informations
\makelettertitle

%-------------------------------------------------------------------------------
% LETTER CONTENT
%-------------------------------------------------------------------------------
\begin{cvletter}

\lettersection{About Me}
As a fifth-year Mechanical Engineering student specializing in numerical simulation and data analysis at Centrale Lyon-ENISE, I am writing to express my strong interest in the M2 internship on "Multimodal Approaches for Handling Missing Information in the Analysis of Affective States." My academic background in engineering, combined with my training in artificial intelligence methods and data processing, aligns perfectly with the requirements of this research project on multimodal affective computing.

\lettersection{Why This Internship?}
The challenge of developing robust multimodal models for affective state detection, even with missing or degraded modalities (e.g., voice, facial expressions, physiological signals), represents exactly the type of interdisciplinary AI problem that fuels my passion for computational methods and human-centered applications. The opportunity to explore fusion strategies and missing data handling using deep learning particularly appeals to me for several reasons:
• The project bridges machine learning with real-world applications in health, well-being, and human-machine interaction
• It combines deep learning for multimodal fusion with innovative techniques for data imputation and robustness
• The use of real datasets for affective analysis represents a cutting-edge approach beyond controlled environments
• The possibility to contribute to more resilient AI systems aligns with my career aspirations in intelligent systems

The innovative environment at CESI LINEACT provides an ideal setting for advancing my expertise in multimodal AI.

\lettersection{Why My Profile Fits?}
Throughout my academic journey, particularly through the MIAM (Méthodes d'Intelligence Artificielle pour la Physique) module and my data analytics certifications, I have developed a strong foundation in machine learning and programming that directly applies to this internship:
• \textbf{Deep Learning and Multimodal Methods:} In the MIAM module, I acquired solid skills in deep learning frameworks like PyTorch and TensorFlow, including implementation for signal processing and constrained models adaptable to multimodal data fusion
• \textbf{Data Analysis and Missing Data Handling:} My IBM Data Analyst and Google Advanced Data Analytics certifications have equipped me with techniques for preprocessing multimodal datasets, handling missing values (e.g., imputation strategies), and evaluating model robustness
• \textbf{Programming Skills:} I have substantial experience with Python development for numerical and data-intensive projects, including microstructure analysis and web applications, demonstrating my ability to work with diverse data sources
• \textbf{MATLAB Proficiency:} I have completed multiple MATLAB certifications (Simulink, Power System OnRamp, Computer Vision), enhancing my capabilities in signal processing relevant to affective modalities like facial and physiological signals
• \textbf{Numerical and Analytical Understanding:} My background in Numerical Solid Mechanics provides the foundation to understand complex data interactions and validation in real-world scenarios
• \textbf{Research Experience:} My fatigue analysis project involving numerical validation and my data analytics training have prepared me for rigorous experimentation on model performance and impact analysis

My hands-on experience in Python-based deep learning models, combined with data handling expertise and engineering background, positions me well to contribute to the design, training, and evaluation of multimodal models for affective states in this project.

I am particularly excited about the prospect of working on a project that advances AI robustness for human-centric applications. The opportunity to engage in cutting-edge research at the intersection of deep learning and affective computing is highly motivating.

Thank you for considering my application. I am confident that my technical skills in machine learning and data analysis align well with your requirements, and I would welcome the opportunity to contribute to your team's work on multimodal affective analysis.

\end{cvletter}

%-------------------------------------------------------------------------------
% Print the signature and enclosures with above letter informations
\makeletterclosing

\end{document}