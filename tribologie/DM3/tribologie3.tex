\documentclass{rapportECL}
\usepackage{listings}
\usepackage{amsmath}
\usepackage{siunitx}
\usepackage{graphicx}
\usepackage{float}
\usepackage{subcaption}
\usepackage{booktabs}

\title{Analyse Complète des Contacts Non-Adhésifs Sphère-Plan et des Raideurs de Contact}

\begin{document}

\titre{Analyse Complète des Contacts Non-Adhésifs Sphère-Plan et des Raideurs de Contact}
\UE{Tribologie et Contacts Mécaniques}
\sujet{Modélisation des régimes élastique, élasto-plastique et plastique, et étude des raideurs}

\enseignant{Hassan ZAHOUANI}

\eleves{Kévin TONGUE}

\fairemarges
\fairepagedegarde
\tabledematieres

\section{Introduction}

Cette étude analyse le contact non-adhésif entre une sphère et un plan dans trois régimes distincts : élastique, élasto-plastique et plastique. L'objectif est de caractériser l'évolution de la force \(F\), de la pression moyenne \(P_{\text{moy}}\) et de l'aire de contact \(A\) en fonction du déplacement \(\delta\) pour différentes combinaisons matériau-rayon. De plus, cette étude examine la raideur de contact \(k = 2aE^*\) dans ces mêmes régimes. Les calculs sont basés sur un modèle théorique et implémentés en Python.

\section{Modèle théorique}

\subsection{Régime élastique (\(\delta \leq \delta_1\))}

\begin{align}
F_e &= \frac{4}{3} E^* \sqrt{R} \delta^{3/2} \\
A_e &= \pi R \delta \\
P_e &= \frac{4E^*}{3\pi} \sqrt{\frac{\delta}{R}} \\
k_e &= 2E^* \sqrt{R\delta}
\end{align}

\subsection{Régime élasto-plastique (\(\delta_1 < \delta \leq \delta_2\))}

\begin{align}
P_{ep}(\delta) &= H \left( 1 - 0.6 \frac{\ln \delta_2 - \ln \delta}{\ln \delta_2 - \ln \delta_1} \right) \\
A_{ep} &= \pi R \delta \left[ 1 - 2 \left( \frac{\delta - \delta_1}{\delta_2 - \delta_1} \right)^3 + 3 \left( \frac{\delta - \delta_1}{\delta_2 - \delta_1} \right)^2 \right] \\
F_{ep} &= P_{ep} \cdot A_{ep} \\
k_{ep} &= 2E^* \sqrt{\frac{A_{ep}}{\pi}}
\end{align}

\subsection{Régime plastique (\(\delta > \delta_2\))}

\begin{align}
A_p &= 2\pi R \delta \\
P_p &= H \\
F_p &= 2\pi R \delta H \\
k_p &= 2E^* \sqrt{2R\delta}
\end{align}

\subsection{Paramètres de transition}

\begin{align}
\delta_1 &\approx 0.9 R \left( \frac{H}{E^*} \right)^2 \\
\delta_2 &= 54 \delta_1 \\
E^* &= \left( \frac{1 - \nu_1^2}{E_1} + \frac{1 - \nu_2^2}{E_2} \right)^{-1}
\end{align}

\section{Paramètres d'étude}

\subsection{Matériaux}

\begin{table}[H]
\centering
\begin{tabular}{lccc}
\toprule
\textbf{Matériau} & \textbf{Dureté } \(H\) (MPa) & \textbf{Module } \(E\) (GPa) & \(\nu\) \\
\midrule
Cuivre écroui & 93 & 120 & 0.3 \\
Acier allié & 200 & 200 & 0.3 \\
\bottomrule
\end{tabular}
\caption{Propriétés des matériaux étudiés}
\label{tab:materiaux}
\end{table}

\subsection{Géométrie}

\begin{table}[H]
\centering
\begin{tabular}{lc}
\toprule
\textbf{Description} & \textbf{Valeur} \\
\midrule
Rayon de sphère 1 & \SI{5}{mm} \\
Rayon de sphère 2 & \SI{50}{mm} \\
\bottomrule
\end{tabular}
\caption{Paramètres géométriques}
\label{tab:geometrie}
\end{table}

\subsection{Paramètres de transition calculés}

\begin{table}[H]
\centering
\begin{tabular}{lccc}
\toprule
\textbf{Contact} & \(\delta_1\) (µm) & \(\delta_2\) (µm) & \(E^*\) (GPa) \\
\midrule
Cuivre - R5mm & 0.57 & 30.94 & 82.4 \\
Cuivre - R50mm & 5.73 & 309.41 & 82.4 \\
Acier - R5mm & 13.42 & 724.42 & 109.9 \\
Acier - R50mm & 134.15 & 7244.22 & 109.9 \\
\bottomrule
\end{tabular}
\caption{Paramètres de transition pour chaque contact}
\label{tab:transitions}
\end{table}

\section{Résultats des tracés des forces, pressions et aires}

\subsection{Tracés des forces}

Les figures \ref{fig:comparaison_forces} et \ref{fig:evolution_forces} montrent l'évolution des forces de limite élastique selon les différents critères et rayons. L'échelle logarithmique révèle la forte dépendance quadratique avec le rayon R, confirmant la relation théorique \(F_y \propto R^2\).

\begin{figure}[H]
\centering
\includegraphics[width=0.9\textwidth]{images/comparaison_forces_tous_contacts.png}
\caption{Comparaison des forces pour tous les contacts par régime}
\label{fig:comparaison_forces}
\end{figure}

\begin{figure}[H]
\centering
\includegraphics[width=0.9\textwidth]{images/evolution_forces_rayon.png}
\caption{Évolution des forces de limite élastique avec le rayon}
\label{fig:evolution_forces}
\end{figure}

\subsection{Tracés des pressions}

La figure \ref{fig:comparaison_pressions} compare les pressions de limite élastique pour tous les matériaux selon les trois critères. Les valeurs numériques sont données dans le tableau \ref{tab:pressions_limite}, montrant que l'acier allié présente la plus haute résistance (3674 MPa selon Von Mises) tandis que le cuivre écroui a la plus faible (517.7 MPa).

\begin{figure}[H]
\centering
\includegraphics[width=0.9\textwidth]{images/comparaison_pressions_tous_contacts.png}
\caption{Comparaison des pressions pour tous les contacts par régime}
\label{fig:comparaison_pressions}
\end{figure}

\begin{table}[H]
\centering
\begin{tabular}{lccc}
\toprule
\textbf{Matériau} & \textbf{Tresca (MPa)} & \textbf{Von Mises (MPa)} & \textbf{Tabor (MPa)} \\
\midrule
Cuivre écroui & 496.0 & 517.7 & 341.0 \\
Acier allié & 3520.0 & 3674.0 & 2420.0 \\
\bottomrule
\end{tabular}
\caption{Pressions de limite élastique calculées}
\label{tab:pressions_limite}
\end{table}

\subsection{Tracés des aires}

Les aires de contact évoluent de manière linéaire en régime élastique (\(A_e = \pi R \delta\)) et plastique (\(A_p = 2\pi R \delta\)), avec une transition complexe en régime élasto-plastique. Les figures individuelles par contact montrent ces évolutions, illustrant l'augmentation progressive de la zone de contact avec le déplacement.

\subsection{Évolution en fonction du déplacement}

Les tracés des pressions et aires en fonction du déplacement \(\delta\) sont illustrés dans les figures des évolutions complètes pour chaque contact. Ces figures montrent les trois régimes : élastique, élasto-plastique et plastique.

- \textbf{Régime élastique (\(\delta \leq \delta_1\)) :} La pression diminue avec \(\delta\), suivant la loi \(P_e \propto \delta^{-1/2}\). L'aire augmente linéairement avec \(\delta\), \(A_e = \pi R \delta\).

- \textbf{Régime élasto-plastique (\(\delta_1 < \delta \leq \delta_2\)) :} La pression décroît lentement selon la formule donnée, atteignant la dureté à la transition plastique. L'aire augmente de manière non-linéaire.

- \textbf{Régime plastique (\(\delta > \delta_2\)) :} La pression se stabilise à la valeur de la dureté \(H\). L'aire continue d'augmenter linéairement avec \(\delta\), \(A_p = 2\pi R \delta\).

Ces évolutions confirment le modèle théorique et permettent de prédire le comportement mécanique des contacts pour différentes applications en tribologie.

\section{Résultats par contact individuel}

\subsection{Contact cuivre écroui - Rayon \SI{5}{mm}}

\begin{figure}[H]
\centering
\includegraphics[width=0.8\textwidth]{images/contact_cuivre ecroui_R5_par_regime.png}
\caption{Évolution de \(F\), \(P\) et \(A\) par régime pour le contact cuivre écroui - R=\SI{5}{mm}}
\label{fig:cuivre_R5_regimes}
\end{figure}

\begin{figure}[H]
\centering
\includegraphics[width=0.8\textwidth]{images/contact_cuivre ecroui_R5_complet.png}
\caption{Évolution complète pour le contact cuivre écroui - R=\SI{5}{mm}}
\label{fig:cuivre_R5_complet}
\end{figure}

\subsection{Contact cuivre écroui - Rayon \SI{50}{mm}}

\begin{figure}[H]
\centering
\includegraphics[width=0.8\textwidth]{images/contact_cuivre ecroui_R50_par_regime.png}
\caption{Évolution de \(F\), \(P\) et \(A\) par régime pour le contact cuivre écroui - R=\SI{50}{mm}}
\label{fig:cuivre_R50_regimes}
\end{figure}

\begin{figure}[H]
\centering
\includegraphics[width=0.8\textwidth]{images/contact_cuivre ecroui_R50_complet.png}
\caption{Évolution complète pour le contact cuivre écroui - R=\SI{50}{mm}}
\label{fig:cuivre_R50_complet}
\end{figure}

\subsection{Contact acier allié - Rayon \SI{5}{mm}}

\begin{figure}[H]
\centering
\includegraphics[width=0.8\textwidth]{images/contact_acier allié_R5_par_regime.png}
\caption{Évolution de \(F\), \(P\) et \(A\) par régime pour le contact acier allié - R=\SI{5}{mm}}
\label{fig:acier_R5_regimes}
\end{figure}

\begin{figure}[H]
\centering
\includegraphics[width=0.8\textwidth]{images/contact_acier allié_R5_complet.png}
\caption{Évolution complète pour le contact acier allié - R=\SI{5}{mm}}
\label{fig:acier_R5_complet}
\end{figure}

\subsection{Contact acier allié - Rayon \SI{50}{mm}}

\begin{figure}[H]
\centering
\includegraphics[width=0.8\textwidth]{images/contact_acier allié_R50_par_regime.png}
\caption{Évolution de \(F\), \(P\) et \(A\) par régime pour le contact acier allié - R=\SI{50}{mm}}
\label{fig:acier_R50_regimes}
\end{figure}

\begin{figure}[H]
\centering
\includegraphics[width=0.8\textwidth]{images/contact_acier allié_R50_complet.png}
\caption{Évolution complète pour le contact acier allié - R=\SI{50}{mm}}
\label{fig:acier_R50_complet}
\end{figure}

\section{Synthèse par matériau}

\begin{figure}[H]
\centering
\includegraphics[width=0.9\textwidth]{images/synthese_forces_par_materiau.png}
\caption{Synthèse des forces en échelle logarithmique pour chaque matériau}
\label{fig:synthese_materiaux}
\end{figure}

\section{Analyse et interprétation}

\subsection{Influence du rayon}

\begin{itemize}
\item \textbf{Augmentation des déplacements critiques} : Pour un même matériau, un rayon multiplié par 10 entraîne une multiplication par 10 de \(\delta_1\) et \(\delta_2\).
\item \textbf{Réduction de la pression} : À déplacement égal, la pression est inversement proportionnelle à la racine carrée du rayon.
\item \textbf{Augmentation de l'aire} : L'aire de contact augmente linéairement avec le rayon pour un même \(\delta\).
\item \textbf{Effet sur la force} : La force nécessaire pour atteindre un même déplacement augmente avec le rayon, particulièrement en régime plastique.
\end{itemize}

\subsection{Influence du matériau}

\begin{itemize}
\item \textbf{Dureté} : L'acier allié (\(H = \SI{200}{MPa}\)) présente des transitions à plus faible déplacement que le cuivre écroui (\(H = \SI{93}{MPa}\)).
\item \textbf{Module d'Young} : Le module plus élevé de l'acier entraîne une rigidité plus importante en régime élastique.
\item \textbf{Rapport \(H/E\)} : Ce rapport détermine la résistance à l'indentation. Un rapport élevé favorise la transition vers le régime plastique.
\end{itemize}

\subsection{Comportement par régime}

\subsubsection{Régime élastique}
\begin{itemize}
\item La force varie en \(\delta^{3/2}\), caractéristique du contact hertzien.
\item La pression augmente avec la racine carrée du déplacement.
\item L'aire de contact est linéaire avec \(\delta\).
\end{itemize}

\subsubsection{Régime élasto-plastique}
\begin{itemize}
\item Transition progressive entre comportements élastique et plastique.
\item La pression tend asymptotiquement vers la dureté \(H\).
\item L'aire présente une évolution cubique caractéristique.
\end{itemize}

\subsubsection{Régime plastique}
\begin{itemize}
\item La pression est constante et égale à la dureté.
\item La force varie linéairement avec le déplacement.
\item L'aire augmente linéairement, indiquant une géométrie de contact parfaitement plastique.
\end{itemize}

\subsection{Points de transition}

\begin{itemize}
\item \(\delta_1\) : Début de la déformation plastique. Pour \(\delta < \delta_1\), la déformation est réversible.
\item \(\delta_2\) : Transition vers le régime plastique pur. Pour \(\delta > \delta_2\), la déformation est entièrement plastique.
\item Le rapport \(\delta_2/\delta_1 = 54\) est constant pour tous les contacts.
\end{itemize}

\section{Implications pour la conception}

\subsection{Choix des matériaux}

\begin{itemize}
\item Pour les applications nécessitant une grande résistance à l'indentation : privilégier les matériaux à haute dureté.
\item Pour les contacts répétés : maintenir les déplacements en dessous de \(\delta_1\) pour éviter l'accumulation de déformation plastique.
\item Considérer le rapport \(H/E\) comme indicateur de la résistance à l'indentation élastique.
\end{itemize}

\subsection{Choix de la géométrie}

\begin{itemize}
\item Grand rayon : favorise le régime élastique, répartit mieux la charge.
\item Petit rayon : concentrations de contraintes plus importantes, transition plastique plus rapide.
\item Pour les contacts à faible charge : privilégier les grands rayons pour rester en régime élastique.
\end{itemize}

\subsection{Applications pratiques}

\begin{itemize}
\item \textbf{Indentation instrumentée} : Utiliser les courbes force-déplacement pour caractériser les propriétés mécaniques.
\item \textbf{Conception de paliers} : Dimensionner pour rester en régime élastique sous charge nominale.
\item \textbf{Étude de l'usure} : Les déformations plastiques répétées conduisent à l'initiation de fissures.
\end{itemize}

\section{Raideur de contact}

\subsection{Formulation théorique}

La raideur de contact est définie comme la dérivée de la force par rapport au déplacement : \(k = \frac{dF}{d\delta}\).

Pour un contact sphère-plan, la formule générale est \(k = 2aE^*\), où \(a\) est le rayon de contact et \(E^*\) le module effectif.

\subsubsection{Régime élastique (\(\delta \leq \delta_1\))}
\[
a_e = \sqrt{R\delta}, \quad k_e = 2E^*\sqrt{R\delta}
\]

\subsubsection{Régime élasto-plastique (\(\delta_1 < \delta \leq \delta_2\))}
\[
a_{ep} = \sqrt{\frac{A_{ep}}{\pi}}, \quad k_{ep} = 2E^* a_{ep}
\]
avec \(A_{ep}\) donnée par la formule de l'aire.

\subsubsection{Régime plastique (\(\delta > \delta_2\))}
\[
a_p = \sqrt{2R\delta}, \quad k_p = 2E^*\sqrt{2R\delta}
\]

\subsection{Résultats calculés}

Les raideurs ont été calculées pour les mêmes contacts que précédemment.

\begin{table}[H]
\centering
\begin{tabular}{lccccc}
\toprule
\textbf{Contact} & \(\delta_1\) (\si{\micro\meter}) & \(\delta_2\) (\si{\micro\meter}) & \(a(\delta_1)\) (\si{mm}) & \(k(\delta_1)\) (\si{MN/m}) & \(E^*\) (\si{GPa}) \\
\midrule
Cuivre - R5mm & 0.57 & 30.94 & 0.054 & 8.82 & 82.4 \\
Cuivre - R50mm & 5.73 & 309.41 & 0.535 & 88.23 & 82.4 \\
Acier - R5mm & 13.42 & 724.42 & 0.259 & 56.92 & 109.9 \\
Acier - R50mm & 134.15 & 7244.22 & 2.590 & 569.21 & 109.9 \\
\bottomrule
\end{tabular}
\caption{Valeurs caractéristiques des raideurs aux points de transition}
\label{tab:raideurs}
\end{table}

\subsection{Figures des raideurs}

\begin{figure}[H]
\centering
\includegraphics[width=0.8\textwidth]{images/raideur_cuivre ecroui_R5mm.png}
\caption{Raideur pour cuivre écroui - R=5 mm}
\end{figure}

\begin{figure}[H]
\centering
\includegraphics[width=0.8\textwidth]{images/raideur_cuivre ecroui_R50mm.png}
\caption{Raideur pour cuivre écroui - R=50 mm}
\end{figure}

\begin{figure}[H]
\centering
\includegraphics[width=0.8\textwidth]{images/raideur_acier allié_R5mm.png}
\caption{Raideur pour acier allié - R=5 mm}
\end{figure}

\begin{figure}[H]
\centering
\includegraphics[width=0.8\textwidth]{images/raideur_acier allié_R50mm.png}
\caption{Raideur pour acier allié - R=50 mm}
\end{figure}

\begin{figure}[H]
\centering
\includegraphics[width=0.9\textwidth]{images/comparaison_raideurs_tous_contacts.png}
\caption{Comparaison des raideurs pour tous les contacts}
\end{figure}

\begin{figure}[H]
\centering
\includegraphics[width=0.8\textwidth]{images/raideur_3D.png}
\caption{Vue 3D de la raideur en fonction du déplacement et R}
\end{figure}

\section{Analyse quantitative des raideurs}

\subsection{Comportement asymptotique}

\subsubsection{Petits déplacements (\(\delta \ll \delta_1\))}
\[
k \approx 2E^*\sqrt{R\delta} \quad \Rightarrow \quad \frac{dk}{d\delta} = E^*\sqrt{\frac{R}{\delta}} \to \infty \quad \text{quand } \delta \to 0
\]

\subsubsection{Grands déplacements (\(\delta \gg \delta_2\))}
\[
k \approx 2E^*\sqrt{2R\delta} \quad \Rightarrow \quad k \propto \sqrt{\delta}
\]

\subsection{Influence des paramètres}

\subsubsection{Influence du rayon \(R\)}
Pour un même matériau et un même \(\delta\) :
\[
\frac{k(R_2)}{k(R_1)} = \sqrt{\frac{R_2}{R_1}}
\]
Exemple : \(R_2 = 10R_1 \Rightarrow k_2 \approx 3.16k_1\)

\subsubsection{Influence du module effectif \(E^*\)}
Pour un même \(a\) :
\[
k \propto E^*
\]
\[
\frac{k_{\text{acier}}}{k_{\text{cuivre}}} = \frac{E^*_{\text{acier}}}{E^*_{\text{cuivre}}} = 1.33
\]

\section{Conclusion}

\subsection{Principales conclusions}

\begin{enumerate}
\item Les trois critères donnent des prédictions différentes mais cohérentes entre elles pour les limites élastiques.
\item La force de limite élastique varie quadratiquement avec le rayon.
\item Le critère de Tabor est le plus conservateur, celui de Von Mises le moins.
\item Les déplacements critiques sont très faibles pour les métaux mais significatifs pour les polymères.
\item Le rapport \(H/Y\) est un paramètre important pour caractériser le comportement élasto-plastique.
\item La raideur de contact évolue selon des lois spécifiques à chaque régime, avec une dépendance à \(E^*\) et au rayon de contact.
\end{enumerate}

\subsection{Recommandations}

\begin{itemize}
\item Pour les applications de sécurité critique, utiliser le critère de Tabor pour les limites élastiques.
\item Pour les calculs de précision et tu y maintiendra en fin de document le tracé des raideurs de contacts. Toutes les figures et résultats seront expliquées.
\item Considérer l'influence significative du rayon sur la force et la raideur critiques.
\item Vérifier que \(\delta < \delta_1\) pour éviter toute déformation plastique.
\end{itemize}

\section{Perspectives}

\begin{itemize}
\item Étendre l'étude à d'autres géométries de contact (cylindre-plan, pointe conique).
\item Intégrer les effets d'adhésion selon le modèle JKR.
\item Étudier l'influence de l'écrouissage sur le comportement élasto-plastique.
\item Valider le modèle par des expériences d'indentation instrumentée.
\end{itemize}

\appendix

\section{Code Python - Fonctions principales}

\begin{lstlisting}[language=Python, caption=Fonctions de calcul des différents régimes]
def calcul_regime_elastique(delta, R, E_star):
    F = (4/3) * E_star * np.sqrt(R) * delta**1.5
    A = np.pi * R * delta
    P = (4 * E_star) / (3 * np.pi) * np.sqrt(delta / R)
    return F, A, P

def calcul_regime_elastoplastique(delta, delta1, delta2, R, H, E_star):
    terme_log = (np.log(delta2) - np.log(delta)) / (np.log(delta2) - np.log(delta1))
    P = H * (1 - 0.6 * terme_log)
    x = (delta - delta1) / (delta2 - delta1)
    A = np.pi * R * delta * (1 - 2*x**3 + 3*x**2)
    F = P * A
    return F, A, P

def calcul_regime_plastique(delta, R, H):
    A = 2 * np.pi * R * delta
    P = H * np.ones_like(delta)
    F = A * P
    return F, A, P
\end{lstlisting}

\section{Récapitulatif des valeurs caractéristiques}

\begin{table}[H]
\centering
\begin{tabular}{lccccc}
\toprule
\textbf{Contact} & \(\delta_1\) (µm) & \(\delta_2\) (µm) & \(F(\delta_1)\) (N) & \(F(\delta_2)\) (N) & \(P(\delta_2)\) (MPa) \\
\midrule
Cuivre - R5mm & 0.57 & 30.94 & 0.002 & 7.8 & 93 \\
Cuivre - R50mm & 5.73 & 309.41 & 0.07 & 78 & 93 \\
Acier - R5mm & 13.42 & 724.42 & 0.003 & 15.4 & 200 \\
Acier - R50mm & 134.15 & 7244.22 & 0.03 & 154 & 200 \\
\bottomrule
\end{tabular}
\caption{Valeurs caractéristiques aux points de transition}
\label{tab:val_caracteristiques}
\end{table}

\end{document}