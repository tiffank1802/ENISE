\documentclass{rapportECL}
\usepackage{listings}
\usepackage{amsmath}
\usepackage{siunitx}
\usepackage{graphicx}
\usepackage{float}
\usepackage{booktabs}
\usepackage{longtable}
\usepackage{array}
\usepackage{caption}
\usepackage{subcaption}
\usepackage{hyperref}
\usepackage{xcolor}

\title{Analyse du Contact Élastique Sphère-Plan et Calcul des Contraintes}

\begin{document}

\titre{Analyse du Contact Élastique Sphère-Plan et Calcul des Contraintes}
\UE{Tribologie et Contacts Mécaniques}
\sujet{Contact Hertzien et Calcul des Contraintes}

\enseignant{Hassan {ZAHOUANI}}

\eleves{
        {Kevin TONGUE}}

\fairemarges
\fairepagedegarde
\tabledematieres

\section{Introduction}

Ce rapport présente une analyse complète du contact élastique entre une sphère en acier et différents types de matériaux plans, selon la théorie de Hertz. L'étude porte sur 12 cas distincts, combinant deux rayons de sphère, deux niveaux de force et trois types de matériaux. Pour chaque cas, les paramètres suivants sont calculés et analysés :

\begin{itemize}
    \item Profil de pression hertzien $p(r)$
    \item Rayon de contact $a$
    \item Enfoncement $\delta$
    \item Pression maximale $p_0$
    \item Raideur de contact $K$
    \item Hauteur du bourlet $h$
\end{itemize}

Cette analyse paramétrique permet de comprendre l'influence de chaque variable sur le comportement du contact et d'identifier les configurations optimales pour différentes applications.

\section{Données du problème}

\subsection{Sphère en acier}

\begin{itemize}
    \item Module d'Young : $E_1 = \SI{210}{GPa}$
    \item Coefficient de Poisson : $\nu_1 = 0.3$
    \item Rayons étudiés : $R = \SI{5}{mm}$ et $R = \SI{100}{mm}$
\end{itemize}

\subsection{Forces appliquées}

\begin{itemize}
    \item $F_1 = \SI{50}{N}$ (charge légère)
    \item $F_2 = \SI{500}{N}$ (charge importante)
\end{itemize}

\subsection{Matériaux des plans}

\begin{table}[H]
\centering
\begin{tabular}{llcc}
\toprule
\textbf{Nom} & \textbf{Matériau} & $E$ (\si{GPa}) & $\nu$ \\
\midrule
Acier & Acier & 210 & 0.3 \\
Fonte & Fonte & 100 & 0.3 \\
Alu & Aluminium & 10 & 0.45 \\
\bottomrule
\end{tabular}
\caption{Matériaux des plans étudiés}
\end{table}

\section{Formules de Hertz}

Les équations fondamentales du contact hertzien sont :

\subsection{Module effectif}

\begin{equation}
\frac{1}{E^*} = \frac{1-\nu_1^2}{E_1} + \frac{1-\nu_2^2}{E_2}
\end{equation}

Pour un plan rigide ($E_2 \to \infty$) :
\begin{equation}
E^* = \frac{E_1}{1-\nu_1^2}
\end{equation}

\subsection{Paramètres de contact}

\begin{align}
& \text{Rayon de contact :} \quad a = \left( \frac{3PR}{4E^*} \right)^{\frac{1}{3}} \\
& \text{Enfoncement :} \quad \delta = \frac{a^2}{R} \\
& \text{Pression maximale :} \quad p_0 = \frac{3P}{2\pi a^2} \\
& \text{Raideur de contact :} \quad K = 2aE^* \\
& \text{Hauteur du bourlet :} \quad h = 0.42 \sqrt{\frac{\delta}{R}}
\end{align}

\subsection{Profil de pression}

\begin{equation}
p(r) = p_0 \sqrt{1 - \frac{r^2}{a^2}} \quad \text{pour } 0 \leq r \leq a
\end{equation}

\section{Résultats des calculs}

\subsection{Tableau récapitulatif des 12 cas}

\begin{table}[H]
\centering
\fontsize{8}{9}\selectfont
\begin{tabular}{|c|c|c|c|c|c|c|}
\hline
$R$ (mm) & $F$ (N) & Plan & $E^*$ (GPa) & $a$ (mm) & $\delta$ ($\mu$m) & $p_0$ (MPa) \\
\hline
5 & 50 & Acier & 115.38 & 0.118 & 2.76 & 1727.2 \\
5 & 50 & Fonte & 74.44 & 0.136 & 3.70 & 1289.6 \\
5 & 50 & Alu & 11.89 & 0.251 & 12.57 & 379.7 \\
\hline
5 & 500 & Acier & 115.38 & 0.253 & 12.83 & 3721.1 \\
5 & 500 & Fonte & 74.44 & 0.293 & 17.19 & 2778.4 \\
5 & 500 & Alu & 11.89 & 0.540 & 58.37 & 818.0 \\
\hline
100 & 50 & Acier & 115.38 & 0.319 & 1.02 & 234.4 \\
100 & 50 & Fonte & 74.44 & 0.369 & 1.36 & 175.0 \\
100 & 50 & Alu & 11.89 & 0.681 & 4.63 & 51.5 \\
\hline
100 & 500 & Acier & 115.38 & 0.688 & 4.73 & 505.0 \\
100 & 500 & Fonte & 74.44 & 0.796 & 6.33 & 377.1 \\
100 & 500 & Alu & 11.89 & 1.466 & 21.50 & 111.0 \\
\hline
\end{tabular}
\caption{Résultats complets des 12 cas d'étude}
\end{table}

\subsection{Hauteurs de bourlet}

\begin{table}[H]
\centering
\fontsize{9}{10}\selectfont
\begin{tabular}{|c|c|c|c|c|}
\hline
$R$ (mm) & $F$ (N) & Acier & Fonte & Alu \\
\hline
5 & 50 & 0.987 $\mu$m & 1.142 $\mu$m & 21.063 $\mu$m \\
5 & 500 & 2.127 $\mu$m & 2.462 $\mu$m & 45.378 $\mu$m \\
100 & 50 & 0.134 $\mu$m & 0.155 $\mu$m & 2.859 $\mu$m \\
100 & 500 & 0.289 $\mu$m & 0.334 $\mu$m & 6.159 $\mu$m \\
\hline
\end{tabular}
\caption{Hauteurs de bourlet pour chaque configuration}
\end{table}

\section{Visualisations}

\subsection{Profils de pression hertziens}

\begin{figure}[H]
    \centering
    \includegraphics[width=0.95\textwidth]{profils_pression_12_cas.png}
    \caption{Profils de pression pour les 12 cas}
    \label{fig:profils_pression}
\end{figure}

La figure \ref{fig:profils_pression} présente les 12 profils de pression hertsiens. On observe que :

\begin{itemize}
    \item La pression maximale $p_0$ est atteinte au centre du contact ($r=0$)
    \item Le profil est semi-elliptique et s'annule à $r=a$
    \item Le rayon de contact $a$ dépend du module effectif et du rayon de la sphère
    \item Les plans plus rigides génèrent des pressions plus élevées
\end{itemize}

\subsection{Raideurs de contact}

\begin{figure}[H]
    \centering
    \includegraphics[width=0.9\textwidth]{raideurs_contact_12_cas.png}
    \caption{Raideurs de contact - Vue 3D et comparaison}
    \label{fig:raideurs_contact}
\end{figure}

La figure \ref{fig:raideurs_contact} montre que la raideur de contact augmente avec :
\begin{itemize}
    \item La force appliquée
    \item Le rayon de la sphère
    \item Le module d'Young du plan
\end{itemize}

\subsection{Hauteurs des bourlets}

\begin{figure}[H]
    \centering
    \includegraphics[width=0.9\textwidth]{hauteurs_bourlets_12_cas.png}
    \caption{Hauteurs des bourlets - Heatmaps et comparaisons}
    \label{fig:hauteurs_bourlets}
\end{figure}

Les heatmaps de la figure \ref{fig:hauteurs_bourlets} permettent de visualiser rapidement l'influence des paramètres sur la hauteur du bourlet.

\subsection{Synthèse comparative}

\begin{figure}[H]
    \centering
    \includegraphics[width=0.95\textwidth]{synthese_comparative.png}
    \caption{Synthèse comparative des résultats}
    \label{fig:synthese}
\end{figure}

La figure \ref{fig:synthese} présente :
\begin{enumerate}
    \item \textbf{Barres des rayons de contact} pour $F=50$N
    \item \textbf{Barres des pressions maximales} pour $F=500$N
    \item \textbf{Courbes force-enfoncement} pour $R=5$mm
    \item \textbf{Matrice de corrélation} entre les paramètres
    \item \textbf{Diagramme radar} pour comparer les configurations
\end{enumerate}

\section{Analyse des résultats}

\subsection{Influence du rayon de la sphère}

\begin{itemize}
    \item Le rayon de contact $a$ augmente avec $R$ : $a \propto R^{1/3}$
    \item La pression maximale $p_0$ diminue lorsque $R$ augmente : $p_0 \propto R^{-2/3}$
    \item L'enfoncement $\delta$ diminue avec $R$ : $\delta \propto R^{-1/3}$
    \item La raideur $K$ augmente avec $R$ : $K \propto R^{2/3}$
\end{itemize}

Exemple : Pour $F=50$N et l'acier, en passant de $R=5$mm à $R=100$mm :
\begin{itemize}
    \item $a$ passe de 0.118 mm à 0.319 mm (facteur 2.7)
    \item $p_0$ passe de 1727.2 MPa à 234.4 MPa (facteur 7.4)
    \item $\delta$ passe de 2.76 $\mu$m à 1.02 $\mu$m (facteur 0.37)
\end{itemize}

\subsection{Influence de la force appliquée}

\begin{itemize}
    \item Le rayon de contact varie comme $P^{1/3}$
    \item L'enfoncement varie comme $P^{2/3}$
    \item La relation force-enfoncement est non-linéaire : $F \propto \delta^{3/2}$
\end{itemize}

Exemple : Pour $R=5$mm et l'acier, en passant de $F=50$N à $F=500$N :
\begin{itemize}
    \item Le rayon passe de 0.118 mm à 0.253 mm (facteur 2.14)
    \item L'enfoncement passe de 2.76 $\mu$m à 12.83 $\mu$m (facteur 4.65)
    \item La pression passe de 1727.2 MPa à 3721.1 MPa (facteur 2.15)
\end{itemize}

\subsection{Influence du matériau du plan}

\begin{itemize}
    \item Un plan plus rigide ($E$ élevé) augmente $p_0$ et $K$
    \item Un plan plus mou ($E$ faible) augmente $a$ et $\delta$
    \item Le module effectif $E^*$ est le paramètre clé
\end{itemize}

Comparaison pour $R=5$mm, $F=50$N :
\begin{itemize}
    \item Acier : $p_0 = 1727.2$ MPa, $a = 0.118$ mm
    \item Fonte : $p_0 = 1289.6$ MPa, $a = 0.136$ mm
    \item Alu : $p_0 = 379.7$ MPa, $a = 0.251$ mm
\end{itemize}

\subsection{Matrice de corrélation}

La matrice de corrélation montre les relations entre les paramètres :

\begin{itemize}
    \item $a$ et $\delta$ sont fortement corrélés (+0.97)
    \item $a$ et $p_0$ sont corrélés négativement (-0.71)
    \item $K$ et $a$ sont corrélés positivement (+0.85)
    \item $h$ est corrélé avec $\delta$ et $a$
\end{itemize}

\section{Conclusion}

Cette analyse paramétrique complète du contact hertzien a permis d'étudier 12 configurations différentes. Les résultats principaux sont :

\begin{enumerate}
    \item \textbf{Influence du rayon} : Les grands rayons réduisent significativement les pressions de contact, ce qui explique l'utilisation de corps roulants de grand diamètre dans les roulements.
    
    \item \textbf{Non-linéarité} : La relation force-enfoncement $F \propto \delta^{3/2}$ est une caractéristique fondamentale des contacts hertsiens.
    
    \item \textbf{Matériau du plan} : Le choix du matériau influence directement les pressions et les raideurs de contact.
    
    \item \textbf{Applications} : Ces résultats sont essentiels pour la conception des roulements, engrenages, cames et autres composants mécaniques soumis à des contacts ponctuels.
\end{enumerate}

\vfill

% \section*{Annexe : Fichiers générés}

% Les fichiers suivants ont été générés par le script Python :

% \begin{itemize}
%     \item \texttt{profils\_pression\_12\_cas.png} : 12 profils de pression
%     \item \texttt{raideurs\_contact\_12\_cas.png} : Raideurs en 3D et 2D
%     \item \texttt{hauteurs\_bourlets\_12\_cas.png} : Heatmaps et comparaisons
%     \item \texttt{synthese\_comparative.png} : Synthèse avec matrice et radar
%     \item \texttt{resultats\_12\_cas.csv} : Tableau des résultats en CSV
%     \item \texttt{hertz\_12\_cas.py} : Script Python complet
% \end{itemize}

% \clearpage
\end{document}
