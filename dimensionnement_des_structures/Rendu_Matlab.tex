\documentclass{rapportECL}
\usepackage{lipsum}
\usepackage{overpic}
\usepackage{xcolor}
\usepackage{graphicx}
\usepackage{tcolorbox}
\title{Rapport ECL - Template} %Titre du fichier

\begin{document}

%----------- Informations du rapport ---------

\titre{Rendus TP-Matlab} %Titre du fichier .pdf
\UE{Méthode des Eléments Finis} %Nom de la UE
\sujet{ Rendus TP  } %Nom du sujet

\enseignant{Feulvarch \textsc{ Eric }} %Nom de l'enseignant

\eleves{Kevin \textsc{TONGUE}\\ BIAOU \textsc{Adebayo Landry} }
\UEU{ENISE-5GM}

%----------- Initialisation -------------------
        
\fairemarges %Afficher les marges
\fairepagedegarde %Créer la page de garde
\tabledematieres %Créer la table de matières

%------------ Corps du rapport ----------------

\section{Construction de la matrice de rigidité globale $[K]$}

La structure de l'exercice est modélisée avec des ressorts de raideur $K = 1\ \text{N·m}^{-1}$.  
Alors nous obtenons la matrice ci-après sur matlab par assemblage:

\begin{figure}[h!]
    \centering
    \begin{minipage}{1\textwidth} %
        \includegraphics[width=\linewidth]{images/Matlab/image1.png} % Remplace par le chemin de ton premier logo
    \end{minipage}\hfill % Espace entre 
    \caption{Matrice de raideur}
\end{figure}

\section{Calcul des valeurs propres de $[K]$}

 Pour le calcul des valeurs propres, nous avons utilisé la fonction \textit{eig} sous \textsc{Matlab} afin de les afficher. Nous avons constaté que les quatre premières valeurs étaient nulles. Cela s'explique par la présence de degrés de liberté associés aux \textbf{translations} et aux \textbf{rotations}, traduisant un \textbf{manque de raideur verticale} dans le système. 
\\ 

Pour remédier à ce problème, il est possible d'introduire une \textbf{pénalité} afin de limiter ces mouvements rigides et ainsi éviter l'apparition de valeurs propres nulles
Nous avions considére que $P=10^6$ (voir dans le code a la fin du rapport).

\section{Vecteur des forces extérieures $\{F\}$}

La masse $m = 10\ \text{g}$ donne une force qui a été appliqué au niveu du Neoud 3:

\[
F_y = -m g = -0.01 \times 9.81 \approx -0.0981\ \text{N}
\]

Le vecteur $\{F\}$ est construit comme suit :
\begin{itemize}
\item Nœuds chargés : seuls les nœuds portant la masse ont une composante $F_y$ non nulle
\item Exemple : si la masse est au nœud 3, alors $F(6) = -0.0981$
\end{itemize}

\section{Le Programme Matlab}

 \begin{lstlisting}[caption={Code MATLAB pour calculs de déplacement}]
clc;
clear all;
Noeud=0.001*[0 0; 100 0 ;50 0]; % coordonnées des noeuds. Chaque noeud a ses coordonnées en x et en y dans chaque ligne de cette matrice.
El=[1 3; 2 3]; % noeuds constituant les éléments. Chaque ligne constitue un élément. Chaque élément est constitué de noeuds.

FORCE=[0.001:0.0001:0.01];
Uf(1)=0;

    F=zeros(6,1);
    F(6,1)=-.01; % application des forces extérieures

stiffness=1;

U=zeros(6,1);
U(6,1)=-.1;

GRAPH(1)=U(6,1)
epsilon=0.0001;
K=zeros(6,6);
for i=1 :size(El,1)
    i1=(El(i,1)-1)*2+1;
    i2=(El(i,1)-1)*2+2;
    i3=(El(i,2)-1)*2+1;
    i4=(El(i,2)-1)*2+2;
    indice=[i1 i2 i3 i4];

    x1=Noeud(El(i,1),1)+U(i1,1); % On choisit la première composante autrement dit la composante en x du noeud correspondant à l'élément
    y1=Noeud(El(i,1),2)+U(i2,1);
    x2=Noeud(El(i,2),1)+U(i3,1);
    y2=Noeud(El(i,2),2)+U(i4,1);

    
    h= sqrt((x2-x1)^2+(y2-y1)^2);
   
    cos=(x2-x1)/h;
    sin=(y2-y1)/h;
    Rot=[cos sin;-sin cos];
    R=[Rot zeros(2,2);zeros(2,2) Rot];
    Ke=R'*[1 0 -1 0 ;0 0 0 0;-1 0 1 0 ;0 0 0 0]*R*stiffness;

    
    K(indice,indice)= K(indice,indice)+Ke;

end

% ...le reste du code...
\end{lstlisting}
\section{Les graphes des obtenus}

\begin{figure}[h!]
    \centering
    \begin{minipage}{1\textwidth} %
        \includegraphics[width=\linewidth]{images/Matlab/image.png} % Remplace par le chemin de ton premier logo
    \end{minipage}\hfill % Espace entre 
    \caption{Modèle non linéaire}
\end{figure}

\begin{figure}[h!]
    \centering
    \begin{minipage}{1\textwidth} %
        \includegraphics[width=\linewidth]{images/Matlab/image2.png} 
        \includegraphics[]{}
        % Remplace par le chemin de ton premier logo
    \end{minipage}\hfill % Espace entre 
    \caption{Modèle linéaire}
\end{figure}


\end{document}
