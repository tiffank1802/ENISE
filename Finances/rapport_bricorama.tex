\documentclass{rapportECL}
\usepackage{listings}
\usepackage{booktabs}
\usepackage{array}
\usepackage{longtable}
\usepackage{eurosym}
\usepackage{amsmath}
\usepackage{xcolor}
\usepackage{colortbl}

\title{Étude de Cas : Acquisition de Bricorama France}

\begin{document}

%----------- Informations du rapport ---------

\UE{Finance d'Entreprise}
\sujet{Valorisation DCF et Montage LBO}
\titre{Étude de Cas : Acquisition de Bricorama France}

\enseignant{Vincent \textsc{NOURRISSON}
            }

\eleves{Kevin \textsc{TONGUE} \\}

%----------- Initialisation -------------------
        
\fairemarges
\fairepagedegarde
\tabledematieres

%------------ Corps du rapport ----------------

\section{Présentation de l'entreprise}

\subsection{Identité de Bricorama France}

\textbf{Bricorama France} est une filiale du groupe Les Mousquetaires, qui intègre ses comptes selon la méthode de l'intégration globale. L'entreprise est spécialisée dans la \textbf{distribution de matériaux de construction et de produits de bricolage}.

\begin{itemize}
    \item \textbf{Activité :} Commerce de détail de produits de bricolage en magasin spécialisé
    \item \textbf{Taille :} Entre 250 et 499 salariés (donnée 2023)
    \item \textbf{Implantation :} Siège social à Noisy-le-Grand (93160)
    \item \textbf{Forme juridique :} Société par Actions Simplifiée (SAS)
\end{itemize}

\subsection{Marché et perspectives}

Le marché du bricolage en France est un marché mature mais dynamique, porté par plusieurs facteurs :

\begin{itemize}
    \item La rénovation énergétique des logements (incitations gouvernementales MaPrimeRénov')
    \item Le mouvement "Do It Yourself" (DIY) culturellement ancré en France
    \item Un marché de la réparation et de l'amélioration de l'habitat résilient
\end{itemize}

\subsubsection{Opportunités}

\begin{itemize}
    \item Développement du e-commerce et du click \& collect
    \item Offre renforcée sur les matériaux écologiques et la rénovation énergétique
    \item Optimisation de l'assortiment et de l'expérience client en magasin
\end{itemize}

\subsubsection{Défis}

\begin{itemize}
    \item Concurrence forte des leaders du secteur (Leroy Merlin, Castorama, Brico Dépôt)
    \item Nécessaire adaptation aux nouvelles habitudes de consommation (digitalisation)
    \item Pression sur les marges dans un contexte inflationniste
\end{itemize}

%----------------------------------------------------------
\section{Analyse financière historique (2022-2024)}
%----------------------------------------------------------

L'analyse repose sur les comptes annuels complets publiés et accessibles sur des plateformes d'informations légales (Pappers, Le Figaro Entreprises).

\subsection{Ratios de liquidité et de vulnérabilité financière}

\begin{table}[H]
\centering
\caption{Définition des ratios d'analyse financière}
\begin{tabular}{|p{4cm}|p{5cm}|p{4cm}|}
\hline
\rowcolor{blue!20}
\textbf{Ratio} & \textbf{Formule} & \textbf{Objectif} \\
\hline
Liquidité Générale & $\frac{\text{Actif Circulant}}{\text{Passif Circulant}}$ & $> 1$ \\
\hline
Liquidité Réduite & $\frac{\text{Actif Circ.} - \text{Stocks}}{\text{Passif Circ.}}$ & $> 0.5$ \\
\hline
Autonomie Financière & $\frac{\text{Capitaux Propres}}{\text{Total Passif}}$ & Plus élevé = mieux \\
\hline
Ratio Dettes/EBE & $\frac{\text{Dettes Financières}}{\text{EBE}}$ & $< 3\text{-}4$ ans \\
\hline
Rentabilité économique & $\frac{\text{EBIT}}{\text{Total Actif}}$ & Variable selon secteur \\
\hline
Marge nette & $\frac{\text{Résultat Net}}{\text{CA}}$ & Variable selon secteur \\
\hline
\end{tabular}
\end{table}

\subsection{Interprétation pour le secteur du bricolage}

\begin{itemize}
    \item \textbf{Liquidité générale :} Un ratio $> 1$ indique que l'entreprise peut couvrir ses dettes à court terme. Les distributeurs comme Bricorama ont souvent un ratio élevé du fait de leurs stocks importants.
    \item \textbf{Liquidité réduite :} Ce ratio exclut les stocks, parfois moins liquides. Il donne une vue plus stricte de la capacité à payer rapidement les dettes.
    \item \textbf{Autonomie financière :} Mesure la part de financement apportée par les actionnaires. Un ratio faible signale une dépendance importante aux dettes.
    \item \textbf{Ratio Dettes/EBE :} Indique en combien d'années l'entreprise pourrait rembourser ses dettes avec son EBE actuel. C'est un indicateur clé de vulnérabilité financière.
\end{itemize}

%----------------------------------------------------------
\section{Hypothèses de valorisation}
%----------------------------------------------------------

\subsection{Données de l'année de base (2024)}

Les hypothèses suivantes sont utilisées pour la valorisation DCF :

\begin{table}[H]
\centering
\caption{Données financières de base (Année N = 2024)}
\begin{tabular}{|l|r|l|}
\hline
\rowcolor{blue!20}
\textbf{Élément} & \textbf{Valeur} & \textbf{Unité} \\
\hline
Chiffre d'Affaires (CA) & 500 000 000 & \euro \\
\hline
Excédent Brut d'Exploitation (EBE) & 40 000 000 & \euro \\
\hline
Résultat d'Exploitation (EBIT) & 30 000 000 & \euro \\
\hline
Besoin en Fonds de Roulement (BFR) & 50 000 000 & \euro \\
\hline
Investissements (Capex) & 12 000 000 & \euro \\
\hline
Dettes financières totales & 80 000 000 & \euro \\
\hline
Trésorerie & 10 000 000 & \euro \\
\hline
\end{tabular}
\end{table}

\subsection{Hypothèses de croissance et taux}

\begin{table}[H]
\centering
\caption{Hypothèses de taux pour la valorisation}
\begin{tabular}{|l|r|p{6cm}|}
\hline
\rowcolor{blue!20}
\textbf{Paramètre} & \textbf{Valeur} & \textbf{Justification} \\
\hline
Taux de croissance du volume & 5\% & Croissance saine mais maîtrisée \\
\hline
Inflation sur prix de vente & 5\% & Contexte inflationniste actuel \\
\hline
Inflation sur coûts de production & 5\% & Hypothèse de pass-through complet \\
\hline
\textbf{Croissance nominale du CA} & \textbf{10,25\%} & $(1+5\%) \times (1+5\%) - 1$ \\
\hline
Taux d'actualisation (WACC) & 8\% & Coût moyen pondéré du capital \\
\hline
Taux de croissance long terme ($g_{\infty}$) & 2\% & Croissance perpétuelle prudente \\
\hline
Taux d'impôt sur les sociétés & 25\% & Taux standard en France \\
\hline
Taux dette senior & 5\% & Conditions de marché actuelles \\
\hline
Multiple EBE pour dette max & 3,5x & Norme bancaire LBO \\
\hline
Horizon de projection & 5 ans & Standard pour valorisation DCF \\
\hline
\end{tabular}
\end{table}

\subsection{Ratios structurels calculés}

À partir des données de base, les ratios suivants sont déterminés :

\begin{table}[H]
\centering
\caption{Ratios structurels}
\begin{tabular}{|l|c|l|}
\hline
\rowcolor{blue!20}
\textbf{Ratio} & \textbf{Valeur} & \textbf{Formule} \\
\hline
Marge EBE & 8,00\% & $\frac{\text{EBE}}{\text{CA}} = \frac{40}{500}$ \\
\hline
Marge EBIT & 6,00\% & $\frac{\text{EBIT}}{\text{CA}} = \frac{30}{500}$ \\
\hline
Ratio Capex/CA & 2,40\% & $\frac{\text{Capex}}{\text{CA}} = \frac{12}{500}$ \\
\hline
Ratio BFR/CA & 10,00\% & $\frac{\text{BFR}}{\text{CA}} = \frac{50}{500}$ \\
\hline
Dette nette & 70 M\euro & Dettes - Trésorerie = $80 - 10$ \\
\hline
\end{tabular}
\end{table}

%----------------------------------------------------------
\section{Valorisation par la méthode DCF}
%----------------------------------------------------------

\subsection{Méthodologie}

La méthode des Discounted Cash Flows (DCF) consiste à :
\begin{enumerate}
    \item Projeter les Flux de Trésorerie Disponibles (FTD ou FCF) sur un horizon explicite (5 ans)
    \item Calculer une Valeur Terminale représentant la valeur au-delà de l'horizon
    \item Actualiser tous ces flux au taux WACC pour obtenir la Valeur d'Entreprise
\end{enumerate}

\subsection{Calcul du Flux de Trésorerie Disponible (FCF)}

Le FCF représente le cash généré par l'entreprise après tous ses investissements opérationnels :

\begin{equation}
\boxed{\text{FCF} = \text{EBE} - \text{Impôt sur EBIT} - \Delta\text{BFR} - \text{Capex}}
\end{equation}

Où :
\begin{itemize}
    \item \textbf{EBE} : Capacité de l'entreprise à générer du cash de son activité
    \item \textbf{Impôt} : Calculé sur le résultat d'exploitation (EBIT), pas sur l'EBE
    \item \textbf{$\Delta$BFR} : Variation du BFR (consomme du cash si l'entreprise croît)
    \item \textbf{Capex} : Dépenses pour maintenir/développer l'outil de travail
\end{itemize}

\subsection{Projection des flux sur 5 ans}

\begin{table}[H]
\centering
\caption{Tableau des flux de trésorerie prévisionnels (en millions d'\euro)}
\small
\begin{tabular}{|l|r|r|r|r|r|r|}
\hline
\rowcolor{blue!20}
\textbf{Poste} & \textbf{N} & \textbf{N+1} & \textbf{N+2} & \textbf{N+3} & \textbf{N+4} & \textbf{N+5} \\
\hline
Chiffre d'Affaires & 500,0 & 551,3 & 607,7 & 670,0 & 738,7 & 814,4 \\
\hline
Croissance CA (\%) & - & 10,25\% & 10,25\% & 10,25\% & 10,25\% & 10,25\% \\
\hline
EBE (marge 8\%) & 40,0 & 44,1 & 48,6 & 53,6 & 59,1 & 65,2 \\
\hline
EBIT (marge 6\%) & 30,0 & 33,1 & 36,5 & 40,2 & 44,3 & 48,9 \\
\hline
- Impôt (25\%) & 7,5 & 8,3 & 9,1 & 10,1 & 11,1 & 12,2 \\
\hline
- Capex (2,4\%) & 12,0 & 13,2 & 14,6 & 16,1 & 17,7 & 19,5 \\
\hline
BFR (10\% CA) & 50,0 & 55,1 & 60,8 & 67,0 & 73,9 & 81,4 \\
\hline
- $\Delta$BFR & - & 5,1 & 5,6 & 6,2 & 6,9 & 7,6 \\
\hline
\textbf{= FCF} & - & \textbf{17,5} & \textbf{19,3} & \textbf{21,3} & \textbf{23,5} & \textbf{25,9} \\
\hline
\end{tabular}
\end{table}

\subsection{Actualisation et Valeur d'Entreprise}

\subsubsection{Facteurs d'actualisation}

Le facteur d'actualisation pour l'année $t$ est :
\begin{equation}
\text{Facteur}_t = \frac{1}{(1 + \text{WACC})^t} = \frac{1}{(1,08)^t}
\end{equation}

\begin{table}[H]
\centering
\caption{Actualisation des FCF}
\begin{tabular}{|l|c|c|c|c|c|}
\hline
\rowcolor{blue!20}
& \textbf{N+1} & \textbf{N+2} & \textbf{N+3} & \textbf{N+4} & \textbf{N+5} \\
\hline
Facteur d'actualisation & 0,9259 & 0,8573 & 0,7938 & 0,7350 & 0,6806 \\
\hline
FCF actualisé (M\euro) & 16,2 & 16,6 & 16,9 & 17,3 & 17,6 \\
\hline
\end{tabular}
\end{table}

\subsubsection{Valeur Terminale}

La Valeur Terminale représente la valeur de l'entreprise au-delà de l'horizon explicite, selon le modèle de Gordon-Shapiro :

\begin{equation}
\text{VT} = \frac{\text{FCF}_{T+1}}{\text{WACC} - g_{\infty}} = \frac{\text{FCF}_T \times (1 + g_{\infty})}{\text{WACC} - g_{\infty}}
\end{equation}

\begin{align}
\text{FCF}_{N+6} &= 25,9 \times (1 + 2\%) = 26,4 \text{ M\euro} \\
\text{VT (non actualisée)} &= \frac{26,4}{8\% - 2\%} = \frac{26,4}{6\%} = 440,3 \text{ M\euro} \\
\text{VT actualisée} &= \frac{440,3}{(1,08)^5} = 299,6 \text{ M\euro}
\end{align}

\subsubsection{Synthèse de la valorisation}

\begin{table}[H]
\centering
\caption{Synthèse de la Valorisation DCF}
\begin{tabular}{|l|r|l|}
\hline
\rowcolor{blue!20}
\textbf{Élément} & \textbf{Valeur} & \textbf{Formule} \\
\hline
Somme des FCF actualisés (5 ans) & 84,6 M\euro & $\sum_{t=1}^{5} \frac{\text{FCF}_t}{(1,08)^t}$ \\
\hline
Valeur Terminale actualisée & 299,6 M\euro & $\frac{\text{VT}}{(1,08)^5}$ \\
\hline
\rowcolor{green!20}
\textbf{Valeur d'Entreprise (VE)} & \textbf{384,2 M\euro} & VAN + VT actualisée \\
\hline
- Dette nette & 70,0 M\euro & Dettes - Trésorerie \\
\hline
\rowcolor{yellow!30}
\textbf{Valeur des Titres (Equity)} & \textbf{314,2 M\euro} & VE - Dette nette \\
\hline
\end{tabular}
\end{table}

%----------------------------------------------------------
\section{Structure de financement de l'acquisition}
%----------------------------------------------------------

L'acquisition serait réalisée par une \textbf{Holding de reprise} (montage LBO) qui contracterait la dette.

\subsection{Capacité d'endettement}

La capacité d'endettement est déterminée par la règle empirique bancaire :

\begin{equation}
\text{Dette Max} \approx \text{Multiple} \times \text{EBE moyen prévisionnel}
\end{equation}

\begin{table}[H]
\centering
\caption{Calcul de la capacité d'endettement}
\begin{tabular}{|l|r|l|}
\hline
\rowcolor{blue!20}
\textbf{Élément} & \textbf{Valeur} & \textbf{Formule} \\
\hline
EBE moyen prévisionnel (N+1 à N+5) & 54,1 M\euro & Moyenne des EBE projetés \\
\hline
Multiple EBE & 3,5x & Norme bancaire LBO \\
\hline
\textbf{Dette maximale} & \textbf{189,4 M\euro} & $54,1 \times 3,5$ \\
\hline
\end{tabular}
\end{table}

\subsection{Apport en capital minimum}

\begin{equation}
\boxed{\text{Apport Minimum} = \text{MAX}(0, \text{Valeur des Titres} - \text{Dette Max})}
\end{equation}

\begin{align}
\text{Apport Minimum} &= \text{MAX}(0, 314,2 - 189,4) \\
&= \textbf{124,8 M\euro}
\end{align}

\subsection{Cas 1 : Endettement uniquement sur la Holding}

Dans ce scénario, seule la Holding contracte la dette. La cible conserve sa propre structure financière.

\begin{itemize}
    \item Dette Senior de la Holding : 189,4 M\euro
    \item Apport en capital : 124,8 M\euro
    \item Total : 314,2 M\euro\ (= Valeur des Titres)
\end{itemize}

\subsection{Cas 2 : Endettement consolidé avec la cible}

On considère la dette globale (Holding + dette existante de Bricorama). La capacité d'endettement globale est réduite, ce qui \textbf{augmente mécaniquement l'apport en capital minimum}.

\begin{align}
\text{Dette totale existante cible} &= 80 \text{ M\euro} \\
\text{Capacité résiduelle} &= 189,4 - 80 = 109,4 \text{ M\euro} \\
\text{Apport minimum (cas consolidé)} &= 314,2 - 109,4 = \textbf{204,8 M\euro}
\end{align}

%----------------------------------------------------------
\section{Budget de trésorerie de la Holding}
%----------------------------------------------------------

Le budget de trésorerie montre la capacité de la Holding à faire face à ses obligations de remboursement.

\subsection{Paramètres de la dette}

\begin{itemize}
    \item Dette Senior contractée : 189,4 M\euro
    \item Taux d'intérêt : 5\% par an
    \item Durée : 5 ans
    \item Remboursement : linéaire (37,9 M\euro/an)
\end{itemize}

\subsection{Tableau de trésorerie prévisionnel}

\begin{table}[H]
\centering
\caption{Budget de trésorerie de la Holding (en millions d'\euro)}
\small
\begin{tabular}{|l|r|r|r|r|r|}
\hline
\rowcolor{blue!20}
\textbf{Élément} & \textbf{N+1} & \textbf{N+2} & \textbf{N+3} & \textbf{N+4} & \textbf{N+5} \\
\hline
Trésorerie début & 124,8 & 94,9 & 68,2 & 44,7 & 24,5 \\
\hline
+ FCF remonté de la cible & 17,5 & 19,3 & 21,3 & 23,5 & 25,9 \\
\hline
- Intérêts dette (5\%) & 9,5 & 7,6 & 5,7 & 3,8 & 1,9 \\
\hline
- Remboursement principal & 37,9 & 37,9 & 37,9 & 37,9 & 37,9 \\
\hline
\rowcolor{green!20}
\textbf{Trésorerie fin} & \textbf{94,9} & \textbf{68,7} & \textbf{45,9} & \textbf{26,5} & \textbf{10,6} \\
\hline
Dette restante & 151,5 & 113,6 & 75,8 & 37,9 & 0,0 \\
\hline
\end{tabular}
\end{table}

\textbf{Conclusion :} La trésorerie reste \textbf{positive sur l'ensemble de la période}, ce qui valide la faisabilité du montage financier. L'apport en capital de 124,8 M\euro\ est suffisant pour assurer le service de la dette.

%----------------------------------------------------------
\section{Annexes : Relation Finance et Ingénierie}
%----------------------------------------------------------

\subsection{Valoriser les compétences d'ingénieur via la finance}

En tant qu'ingénieur, la compréhension des principes financiers apporte une vision transverse :

\begin{enumerate}
    \item \textbf{Traduire technique en économique :} Chiffrer la rentabilité d'un investissement (nouvelle ligne de production, automatisation) et le justifier financièrement via ROI, VAN, TRI.
    
    \item \textbf{Communication avec la direction financière :} Parler le même langage. Comprendre pourquoi un projet peut être rejeté pour des raisons de trésorerie ou de BFR, même s'il est techniquement pertinent.
    
    \item \textbf{Optimisation globale :} Proposer des solutions qui réduisent les coûts (production, approvisionnement) ou optimisent le BFR (gestion des stocks, délais clients), avec un impact direct sur la trésorerie et la valeur de l'entreprise.
\end{enumerate}

\subsection{Critères de décision pour un investissement significatif}

Pour un investissement majeur (agrandissement d'entrepôt, nouveau système logistique), les critères clés sont :

\begin{table}[H]
\centering
\caption{Critères de décision d'investissement}
\begin{tabular}{|l|p{8cm}|}
\hline
\rowcolor{blue!20}
\textbf{Critère} & \textbf{Description} \\
\hline
TRI (Taux de Rentabilité Interne) & Doit être supérieur au WACC (ici 8\%) \\
\hline
VAN (Valeur Actuelle Nette) & Doit être positive pour créer de la valeur \\
\hline
Délai de récupération (Payback) & Temps pour rembourser l'investissement \\
\hline
Impact stratégique & Parts de marché, différenciation, réglementation \\
\hline
Flexibilité et risque & Modularité, risques de dépassement de coûts \\
\hline
\end{tabular}
\end{table}

\subsection{Synthèse des relations Finance / Opérationnel}

\textbf{Points de friction potentiels :}
\begin{itemize}
    \item Les opérationnels trouvent les financiers trop "contraints" (procédures, budgets rigides)
    \item Les financiers trouvent les opérationnels trop "optimistes" sur leurs prévisions
\end{itemize}

\textbf{Pratiques vertueuses :}
\begin{itemize}
    \item Réunions régulières de budgétisation et suivi mensuel (reporting)
    \item Tableaux de bord partagés (KPIs financiers et opérationnels)
    \item Implication des financiers dès le départ dans les projets d'investissement
\end{itemize}

\textbf{Rôle de l'ingénieur :} Interface idéale entre technique et finance, capable de traduire les aspects techniques en termes financiers et d'expliquer les contraintes financières aux équipes opérationnelles.

%----------------------------------------------------------
\section{Conclusion}
%----------------------------------------------------------

Cette étude de cas sur l'acquisition de Bricorama France a permis de mettre en pratique les concepts fondamentaux de la finance d'entreprise :

\begin{enumerate}
    \item \textbf{Analyse financière :} Compréhension des ratios de liquidité et de vulnérabilité financière pour évaluer la santé de l'entreprise cible.
    
    \item \textbf{Valorisation DCF :} Application de la méthode des flux de trésorerie actualisés pour déterminer la valeur d'entreprise (384,2 M\euro) et la valeur des titres (314,2 M\euro).
    
    \item \textbf{Montage LBO :} Structuration du financement avec une dette senior (189,4 M\euro) et un apport en capital minimum (124,8 M\euro).
    
    \item \textbf{Budget de trésorerie :} Validation de la faisabilité du montage avec une trésorerie positive sur toute la durée du remboursement.
\end{enumerate}

La maîtrise de ces outils financiers est essentielle pour un ingénieur souhaitant participer activement aux décisions stratégiques de l'entreprise et dialoguer efficacement avec les directions financières.

%----------------------------------------------------------
\section*{Sources}
%----------------------------------------------------------

\begin{itemize}
    \item Pappers - Comptes annuels Bricorama France (2022-2024)
    \item Le Figaro Entreprises - Informations légales Bricorama
    \item Données de marché du secteur du bricolage en France
\end{itemize}

\end{document}
