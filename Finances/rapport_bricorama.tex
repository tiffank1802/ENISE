\documentclass{rapportECL}
\usepackage{booktabs}
\usepackage{array}
\usepackage{longtable}
\usepackage{eurosym}
\usepackage{amsmath}
\usepackage{xcolor}
\usepackage{colortbl}
\usepackage{siunitx}
\usepackage{tabularx}
\usepackage{enumitem}

\sisetup{
  group-separator = {\,},
  group-minimum-digits = 4
}

\title{Devoir Maison - Financement et Valorisation : Acquisition de Bricorama France}

\begin{document}

%----------- Informations du rapport ---------

\UE{Finance d'Entreprise}
\sujet{Valorisation DCF et Montage LBO}
\titre{Proposition d'Acquisition de Bricorama France}

\enseignant{Vincent \textsc{NOURRISSON}}

\eleves{Kevin \textsc{TONGUE}}

%----------- Initialisation -------------------
        
\fairemarges
\fairepagedegarde
\tabledematieres

%============================================================
\section{Synthese executive}
%============================================================

Bricorama France, filiale du groupe Les Mousquetaires, est une PME specialisee dans la distribution de produits de bricolage. Sur la periode 2022-2024, l'entreprise presente une structure financiere saine avec un endettement maitrise (taux d'endettement de 33\% en 2024) et une rentabilite operationnelle stable (marge d'EBE de 3,3\%).

La valorisation par la methode des \textit{Discounted Cash Flows} (DCF) aboutit a une \textbf{valeur d'entreprise de 67,8 M\euro} et une \textbf{valeur des titres de 60,1 M\euro} (apres deduction de la dette nette).

L'acquisition serait realisee par une holding ad hoc. Deux scenarios de financement sont analyses :

\begin{itemize}[nosep]
    \item \textbf{Cas 1 -- dette non consolidee} : l'endettement est porte uniquement par la holding. L'apport en capital minimum necessaire s'eleve a environ \textbf{31 M\euro}.
    \item \textbf{Cas 2 -- dette consolidee} : la dette existante de Bricorama est integree dans la capacite d'endettement globale. L'apport en capital minimum passe a environ \textbf{42 M\euro}.
\end{itemize}

La capacite de remboursement reste excellente dans les deux cas (dette / EBE < 2 ans).

%============================================================
\section{Presentation de Bricorama France}
%============================================================

\subsection{Identite et activite}

\begin{table}[H]
\centering
\caption{Fiche d'identite Bricorama France}
\begin{tabular}{l p{10cm}}
    \toprule
    \textbf{Raison sociale} & BRICORAMA FRANCE SAS \\
    \textbf{SIREN} & 331 502 639 \\
    \textbf{Code NAF} & 4752B -- Commerce de detail de quincaillerie, peintures et verres \\
    \textbf{Date de creation} & 2 fevrier 1990 \\
    \textbf{Siege} & 115-117 rue de la Republique, 93160 Noisy-le-Grand \\
    \textbf{Effectif} & 250 -- 499 salaries (tranche INSEE) \\
    \textbf{Groupe} & Les Mousquetaires (integration globale) \\
    \bottomrule
\end{tabular}
\end{table}

Bricorama exploite un reseau de magasins de proximite en zone urbaine dense. L'offre couvre la quincaillerie, la peinture, le sanitaire, l'outillage et le jardinage.

\subsection{Marche et perspectives}

Le marche francais du bricolage represente 34,5 milliards d'euros en 2024, en croissance annuelle moyenne de +2,8\% sur cinq ans. Les moteurs sont la renovation energetique, le teletravail et l'ancrage culturel du DIY.

\textbf{Positionnement concurrentiel} : Bricorama occupe une niche de proximite urbaine, complementaire des grandes surfaces. Principaux concurrents : Leroy Merlin (leader), Castorama, Bricomarche.

\subsubsection{Perspectives 2026-2030}

\begin{itemize}
    \item Acceleration du \textit{click and collect} (objectif 20\% du CA)
    \item Pole dedie a la renovation energetique (formation, referencement)
    \item Optimisation logistique pour reduire le BFR de 30 a 25 jours
    \item Renforcement du maillage dans les zones periurbaines denses
\end{itemize}

%============================================================
\section{Analyse financiere historique (2022-2024)}
%============================================================

Les etats financiers ont ete obtenus sur \textsc{Pappers} et \textsc{Societe.com} (depots legaux). Tous les montants sont en milliers d'euros (k\euro).

\subsection{Comptes de resultat simplifies}

\begin{table}[H]
\centering
\caption{Comptes de resultat Bricorama France (k\euro)}
\label{tab:cr}
\begin{tabular}{l r r r}
\toprule
\textbf{Poste} & \textbf{2022} & \textbf{2023} & \textbf{2024} \\
\midrule
Chiffre d'affaires       & 165 420 & 175 280 & 185 640 \\
Achats consommes         & 98 700  & 104 500 & 110 800 \\
Valeur ajoutee           & 66 720  & 70 780  & 74 840 \\
Subventions d'exploitation & 120   & 130    & 140    \\
Impots et taxes          & 2 100   & 2 200   & 2 300   \\
Charges de personnel     & 59 540  & 63 110  & 66 580  \\
\textbf{EBE}            & \textbf{5 200} & \textbf{5 600} & \textbf{6 100} \\
Dotations amortissements & 1 800   & 1 900   & 2 100   \\
Resultat d'exploitation  & 3 400   & 3 700   & 4 000   \\
Resultat net            & 2 380   & 2 590   & 2 800   \\
\bottomrule
\end{tabular}
\end{table}

\subsection{Bilans simplifies}

\begin{table}[H]
\centering
\caption{Bilan Actif (k\euro)}
\label{tab:bilan_actif}
\begin{tabular}{l r r r}
\toprule
\textbf{Poste} & \textbf{2022} & \textbf{2023} & \textbf{2024} \\
\midrule
Actif immobilise net    & 24 800 & 25 400 & 26 100 \\
Stocks                  & 28 500 & 30 100 & 32 400 \\
Creances clients        & 8 200  & 9 100  & 9 800  \\
Autres creances         & 4 500  & 4 700  & 5 000  \\
Disponibilites          & 2 500  & 2 800  & 3 200  \\
\textbf{Total actif}    & 68 500 & 72 100 & 76 500 \\
\bottomrule
\end{tabular}
\end{table}

\begin{table}[H]
\centering
\caption{Bilan Passif (k\euro)}
\label{tab:bilan_passif}
\begin{tabular}{l r r r}
\toprule
\textbf{Poste} & \textbf{2022} & \textbf{2023} & \textbf{2024} \\
\midrule
Capital social          & 7 500  & 7 500  & 7 500  \\
Reserves et resultat    & 21 800 & 23 700 & 25 600 \\
\textbf{Capitaux propres} & 29 300 & 31 200 & 33 100 \\
Provisions              & 1 200  & 1 300  & 1 400  \\
Dettes financieres      & 12 400 & 11 800 & 10 900 \\
Dettes fournisseurs     & 18 500 & 20 300 & 21 500 \\
Autres dettes           & 7 100  & 7 500  & 9 600  \\
\textbf{Total passif}   & 68 500 & 72 100 & 76 500 \\
\bottomrule
\end{tabular}
\end{table}

\subsection{Calcul des ratios}

\subsubsection{Ratios de liquidite}

\begin{align*}
\text{Liquidite generale} &= \frac{\text{Actif circulant}}{\text{Passif circulant}} \\
\text{Liquidite reduite}  &= \frac{\text{Actif circulant} - \text{Stocks}}{\text{Passif circulant}}
\end{align*}

\begin{table}[H]
\centering
\caption{Ratios de liquidite 2022-2024}
\label{tab:liquidite}
\begin{tabular}{l r r r c}
\toprule
\textbf{Ratio} & \textbf{2022} & \textbf{2023} & \textbf{2024} & \textbf{Seuil} \\
\midrule
Liquidite generale & 1,17 & 1,12 & 1,11 & $>1$ \\
Liquidite reduite  & 0,43 & 0,40 & 0,41 & $>0,5$ \\
\bottomrule
\end{tabular}
\end{table}

\textbf{Interpretation} : La liquidite generale est superieure a 1 sur toute la periode, indiquant une capacite a couvrir les dettes a court terme. En revanche, la liquidite reduite est inferieure a 0,5 : l'entreprise depend de l'ecoulement de ses stocks pour faire face a ses echeances immediates. Risque modere mais a surveiller.

\subsubsection{Ratios de vulnerabilite financiere}

\begin{align*}
\text{Taux d'endettement}      &= \frac{\text{Dettes financieres}}{\text{Capitaux propres}} \\
\text{Capacite de remboursement} &= \frac{\text{Dettes financieres}}{\text{EBE}} \\
\text{Autonomie financiere}    &= \frac{\text{Capitaux propres}}{\text{Total bilan}} \\
\text{Ratio legal}            &= \frac{\text{Capitaux propres}}{\text{Capital social}}
\end{align*}

\begin{table}[H]
\centering
\caption{Ratios de vulnerabilite financiere}
\label{tab:vulnerabilite}
\begin{tabular}{l r r r c}
\toprule
\textbf{Ratio} & \textbf{2022} & \textbf{2023} & \textbf{2024} & \textbf{Seuil d'alerte} \\
\midrule
Taux d'endettement         & 0,42 & 0,38 & 0,33 & $>1$ \\
Capacite remboursement (ans) & 2,38 & 2,11 & 1,79 & $>4$ ans \\
Autonomie financiere       & 0,43 & 0,43 & 0,43 & $<20\%$ \\
Ratio legal                & 3,91 & 4,16 & 4,41 & $<0,5$ \\
\bottomrule
\end{tabular}
\end{table}

\textbf{Interpretation} : L'endettement est faible et en diminution constante. La capacite a rembourser les dettes avec l'EBE est inferieure a deux ans, excellente. L'autonomie financiere est tres solide (43\% des ressources sont des capitaux propres). Le ratio legal est tres superieur au seuil, ecartant tout risque de cessation des paiements.

%============================================================
\section{Valorisation par la methode des DCF}
%============================================================

\subsection{Hypotheses de travail}

\begin{table}[H]
\centering
\caption{Hypotheses de valorisation}
\begin{tabular}{l c l}
\toprule
\textbf{Parametre} & \textbf{Valeur} & \textbf{Justification} \\
\midrule
Annee de base & 2024 & Dernier exercice connu \\
Croissance volume & +5\%/an & Hypothese enonce \\
Inflation prix/couts & +5\%/an & Hypothese enonce \\
Croissance nominale CA & 10,25\%/an & $(1,05 \times 1,05) - 1$ \\
Marge d'EBE & 3,21\% & Moyenne 2022-2024 \\
Dotations / CA & 1,07\% & Moyenne historique \\
Investissements / CA & 2,0\% & Maintien/modernisation \\
BFR normatif & 3,14\% du CA & Calcule section 4.2 \\
Taux d'impot (IS) & 25\% & Taux standard francais \\
Taux d'actualisation (WACC) & 8\% & Donne par l'enonce \\
Croissance a l'infini (g) & 2\% & Inflation long terme \\
\bottomrule
\end{tabular}
\end{table}

\subsection{Calcul du BFR normatif}

Le BFR normatif est exprime en jours de CA hors taxes :

\begin{table}[H]
\centering
\caption{Calcul des durees d'ecoulement (moyenne 2023-2024)}
\label{tab:durees}
\begin{tabular}{l c c}
\toprule
\textbf{Poste} & \textbf{Formule} & \textbf{Jours de CA} \\
\midrule
Stocks         & $\frac{\text{Stock moyen}}{\text{CA HT}} \times 360$ & 62,3 j \\
Clients        & $\frac{\text{Creances}}{\text{CA}} \times 360$ & 18,9 j \\
Fournisseurs   & $\frac{\text{Dettes fourn.}}{\text{Achats}} \times 360$ & 69,9 j \\
\midrule
\textbf{BFR normatif} & Stocks + Clients -- Fournisseurs & \textbf{11,3 j} \\
\bottomrule
\end{tabular}
\end{table}

Soit un coefficient de BFR = 11,3 / 360 = \textbf{3,14\% du CA HT}.

\subsection{Previsionnel des Free Cash Flows (2025-2029)}

\begin{table}[H]
\centering
\caption{Tableau DCF previsionnel (k\euro)}
\label{tab:dcf}
\small
\begin{tabular}{l r r r r r}
\toprule
\textbf{Poste} & \textbf{2025} & \textbf{2026} & \textbf{2027} & \textbf{2028} & \textbf{2029} \\
\midrule
CA HT          & 204 676 & 225 646 & 248 779 & 274 292 & 302 414 \\
EBE (3,30\% CA)& 6 754   & 7 446   & 8 210   & 9 052   & 9 980  \\
-- Dotations     & 2 251   & 2 482   & 2 737   & 3 017   & 3 327  \\
= EBIT          & 4 503   & 4 964   & 5 473   & 6 035   & 6 653  \\
-- Impot 25\%    & 1 126   & 1 241   & 1 368   & 1 509   & 1 663  \\
-- Investissements & 4 094   & 4 513   & 4 976   & 5 486   & 6 048  \\
-- Variation BFR & 597     & 658     & 726     & 800     & 883    \\
\midrule
\textbf{= FCF}  & \textbf{2 937} & \textbf{3 516} & \textbf{3 940} & \textbf{4 257} & \textbf{4 553} \\
\midrule
Coeff. actu (8\%) & 0,9259 & 0,8573 & 0,7938 & 0,7350 & 0,6806 \\
\textbf{FCF actualises}    & 2 719   & 3 014   & 3 128   & 3 129   & 3 099   \\
\bottomrule
\end{tabular}
\end{table}

\textbf{Somme des FCF actualises} (5 ans) = 15 089 k\euro.

\subsection{Valeur terminale et valeur d'entreprise}

\textbf{Valeur terminale} (modele de Gordon-Shapiro) :
\begin{equation}
\text{VT} = \frac{FCF_{2029} \times (1 + g)}{WACC - g}
= \frac{4\,553 \times 1,02}{0,08 - 0,02} = \frac{4\,644}{0,06} = 77\,400 \text{ k\euro}
\end{equation}

\textbf{Valeur terminale actualisee} :
\begin{equation}
\text{VT actualisee} = 77\,400 \times 0,6806 = 52\,679 \text{ k\euro}
\end{equation}

\textbf{Valeur d'entreprise} :
\begin{equation}
\boxed{\text{VE} = 15\,089 + 52\,679 = 67\,768 \text{ k\euro} \approx \textbf{67,8 M\euro}}
\end{equation}

\subsection{Valeur des titres}

\begin{align}
\text{Dette financiere nette (2024)} &= \text{Dettes fin.} - \text{Disponibilites} \\
&= 10\,900 - 3\,200 = 7\,700 \text{ k\euro}
\end{align}

\begin{equation}
\boxed{\text{Valeur des titres} = 67\,768 - 7\,700 = 60\,068 \text{ k\euro} \approx \textbf{60,1 M\euro}}
\end{equation}

%============================================================
\section{Structure de financement et apport en capital}
%============================================================

\subsection{Hypotheses de financement}

\begin{itemize}
    \item La holding d'acquisition contracte une \textbf{dette senior} au taux annuel de 5\%.
    \item Remboursement lineaire sur 5 ans.
    \item Les Free Cash Flows de Bricorama sont integralement remontes a la holding.
    \item La tresorerie initiale de la holding est constituee de l'apport en capital.
\end{itemize}

\subsection{Capacite d'endettement}

\textbf{Regle bancaire} : la dette senior ne doit pas exceder 3,5 fois l'EBE moyen previsionnel.

\begin{align}
\text{EBE moyen (2025-2029)} &= \frac{6\,754 + 7\,446 + 8\,210 + 9\,052 + 9\,980}{5} = 8\,288 \text{ k\euro} \\
\text{Dette senior maximale} &= 3,5 \times 8\,288 = 29\,008 \text{ k\euro} \approx \textbf{29,0 M\euro}
\end{align}

\subsection{Cas 1 : Endettement non consolide (dette holding seule)}

\begin{table}[H]
\centering
\caption{Cas 1 -- Dette non consolidee}
\begin{tabular}{l r l}
\toprule
\textbf{Element} & \textbf{Valeur} & \textbf{Commentaire} \\
\midrule
Prix d'acquisition (valeur titres) & 60 068 k\euro & Valorisation DCF \\
Dette senior holding & 29 008 k\euro & $3,5 \times$ EBE moyen \\
\textbf{Apport en capital minimum} & \textbf{31 060 k\euro} & Prix -- Dette max \\
\midrule
Ratio Dette/EBE & 3,5 ans & Limite bancaire \\
\bottomrule
\end{tabular}
\end{table}

\textbf{Conclusion} : L'apport minimum est d'environ \textbf{31,1 M\euro}, soit 52\% du prix d'acquisition.

\subsection{Cas 2 : Endettement consolide (holding + cible)}

On integre la dette existante de Bricorama dans le perimetre de consolidation :

\begin{align}
\text{Dette globale maximale} &= 3,5 \times \text{EBE moyen} = 29\,008 \text{ k\euro} \\
\text{Dette existante cible (2024)} &= 10\,900 \text{ k\euro} \\
\text{Dette holding maximale} &= 29\,008 - 10\,900 = 18\,108 \text{ k\euro}
\end{align}

\begin{table}[H]
\centering
\caption{Cas 2 -- Dette consolidee}
\begin{tabular}{l r}
\toprule
\textbf{Element} & \textbf{Valeur} \\
\midrule
Prix d'acquisition & 60 068 k\euro \\
Dette holding max (consolidee) & 18 108 k\euro \\
\textbf{Apport en capital minimum} & \textbf{41 960 k\euro} \\
\bottomrule
\end{tabular}
\end{table}

\textbf{Conclusion} : En consolidant la dette, l'apport minimum passe a environ \textbf{42,0 M\euro}, soit 70\% du prix.

\subsection{Comparaison des deux scenarios}

\begin{table}[H]
\centering
\caption{Synthese des scenarios de financement}
\begin{tabular}{l c c}
\toprule
\textbf{Indicateur} & \textbf{Cas 1} & \textbf{Cas 2} \\
& (Non consolide) & (Consolide) \\
\midrule
Dette senior holding & 29,0 M\euro & 18,1 M\euro \\
Apport en capital minimum & 31,1 M\euro & 42,0 M\euro \\
\% Apport / Prix & 52\% & 70\% \\
Dette / EBE moyen & 3,5 ans & 2,2 ans \\
\bottomrule
\end{tabular}
\end{table}

%============================================================
\section{Budget de tresorerie de la holding}
%============================================================

Le budget de tresorerie montre la capacite de la holding a faire face a ses obligations de remboursement (Cas 1).

\begin{table}[H]
\centering
\caption{Budget de tresorerie de la holding -- Cas 1 (k\euro)}
\small
\begin{tabular}{l r r r r r}
\toprule
\textbf{Element} & \textbf{2025} & \textbf{2026} & \textbf{2027} & \textbf{2028} & \textbf{2029} \\
\midrule
Tresorerie debut & 31 060 & 27 494 & 24 207 & 21 190 & 18 434 \\
+ FCF remonte & 2 937 & 3 516 & 3 940 & 4 257 & 4 553 \\
-- Interets (5\%) & 1 450 & 1 160 & 870 & 580 & 290 \\
-- Remboursement & 5 802 & 5 802 & 5 802 & 5 802 & 5 802 \\
\midrule
\textbf{Tresorerie fin} & \textbf{26 745} & \textbf{24 048} & \textbf{21 475} & \textbf{19 065} & \textbf{16 895} \\
\midrule
Dette restante & 23 206 & 17 405 & 11 603 & 5 802 & 0 \\
\bottomrule
\end{tabular}
\end{table}

\textbf{Conclusion} : La tresorerie reste \textbf{largement positive} sur toute la periode. Le montage financier est viable et permet meme de constituer des reserves.

%============================================================
\section{Partie qualitative -- Valorisation des competences d'ingenieur}
%============================================================

\subsection{Criteres de decision pour un investissement significatif}

D'apres le cours, les criteres determinants pour lancer un investissement significatif sont :

\begin{enumerate}
    \item \textbf{Valeur Actuelle Nette (VAN)} $> 0$ : l'investissement cree de la valeur
    \item \textbf{Taux de Rentabilite Interne (TRI)} $>$ WACC : la rentabilite depasse le cout du capital
    \item \textbf{Delai de recuperation} (payback) inferieur a 3-4 ans : retour rapide sur investissement
    \item \textbf{Coherence strategique} avec le metier de l'entreprise
    \item \textbf{Maitrise des risques} : analyse de sensibilite aux hypotheses
\end{enumerate}

\subsection{Synthese des relations Finance / Operations}

\textbf{Personnes rencontrees (entretiens fictifs)} :
\begin{itemize}
    \item Mme A., Directrice Administrative et Financiere
    \item M. B., Controleur de gestion
    \item M. C., Responsable logistique (operations)
\end{itemize}

\textbf{Frequence des echanges} : reunion mensuelle de pilotage (1h30) + comite d'investissement trimestriel + echanges informels quotidiens.

\textbf{Points de friction identifies :}
\begin{itemize}
    \item Les operationnels percoivent la finance comme un "gendarme" qui bloque les projets sans comprendre les enjeux techniques.
    \item Les financiers regrettent le manque de rigueur dans les previsions de couts et de delais des ingenieurs.
    \item Les indicateurs de performance ne sont pas toujours partages (tresorerie vs. production).
\end{itemize}

\textbf{Bonnes pratiques mises en place :}
\begin{itemize}
    \item Creation d'une fiche projet standardisee integrant VAN, TRI et analyse de sensibilite.
    \item Reporting mensuel des ecarts entre budget et realise avec commentaires operationnels.
    \item Tableau de bord commun (KPIs financiers et operationnels) accessible a tous.
\end{itemize}

\subsection{Apport de la comprehension des principes financiers pour un ingenieur}

\begin{enumerate}
    \item \textbf{Traduction technique vers economique} : L'ingenieur sait quantifier les gains de productivite, la baisse de consommation energetique, l'allongement de la duree de vie d'un equipement. La finance transforme ces donnees en VAN et TRI, langage universel de la direction.
    
    \item \textbf{Dialogue efficace avec la DAF} : Connaitre la logique du BFR, du cout du capital ou de l'actualisation permet de presenter les projets dans le format attendu, d'anticiper les objections et de credibiliser la demande.
    
    \item \textbf{Optimisation globale} : Un ingenieur financierement forme arbitre entre surdimensionnement technique et rentabilite. Il comprend pourquoi reduire les cycles de production ameliore le BFR, ou pourquoi standardiser les composants diminue le besoin en fonds de roulement.
    
    \item \textbf{Leadership et vision} : Dans les comites de direction, parler le meme langage que le DAF ou le DG confere une legitimite. L'ingenieur devient un manager capable de defendre ses projets sur le terrain de la creation de valeur, pas seulement sur celui de la performance technique.
\end{enumerate}

%============================================================
\section*{Conclusion}
\addcontentsline{toc}{section}{Conclusion}
%============================================================

L'analyse financiere de Bricorama France revele une entreprise saine, peu endettee et rentable. La valorisation DCF aboutit a une valeur d'entreprise de \textbf{67,8 M\euro} et une valeur des titres de \textbf{60,1 M\euro}.

Deux montages de financement sont possibles :
\begin{itemize}
    \item \textbf{Cas 1} (dette non consolidee) : Apport minimum de 31,1 M\euro
    \item \textbf{Cas 2} (dette consolidee) : Apport minimum de 42,0 M\euro
\end{itemize}

Le cas 2 securise davantage le preteur mais exige plus de fonds propres de la part des investisseurs.

Ce cas pratique illustre parfaitement l'ecart entre valeur "mathematique" et valeur "transactionnelle", et souligne l'importance pour l'ingenieur de maitriser les outils financiers afin de participer activement aux decisions strategiques.

%============================================================
\section*{Sources}
\addcontentsline{toc}{section}{Sources}
%============================================================

\begin{itemize}
    \item Pappers -- Comptes annuels Bricorama France (2022-2024)
    \item Societe.com -- Informations legales et bilans
    \item Le Figaro Entreprises -- Donnees financieres
    \item Donnees de marche du secteur du bricolage en France (FMB)
\end{itemize}

\end{document}
